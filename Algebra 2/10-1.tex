\documentclass[twoside]{article}
    \setlength{\oddsidemargin}{0.25 in}
    \setlength{\evensidemargin}{-0.25 in}
    \setlength{\topmargin}{-0.6 in}
    \setlength{\textwidth}{6.5 in}
    \setlength{\textheight}{8.5 in}
    \setlength{\headsep}{0.75 in}
    \setlength{\parindent}{0 in}
    \setlength{\parskip}{0.1 in}
    
    %
    % ADD PACKAGES here:
    %
    
    \usepackage{amsmath,amsfonts,amssymb,graphicx,mathtools,flexisym}
    
    %
    % The following commands set up the lecnum (lecture number)
    % counter and make various numbering schemes work relative
    % to the lecture number.
    %
    \newcounter{lecnum}
    \renewcommand{\thepage}{\thelecnum-\arabic{page}}
    \renewcommand{\thesection}{\thelecnum.\arabic{section}}
    \renewcommand{\theequation}{\thelecnum.\arabic{equation}}
    \renewcommand{\thefigure}{\thelecnum.\arabic{figure}}
    \renewcommand{\thetable}{\thelecnum.\arabic{table}}
    \newcommand{\N}{\mathbb{N}}
    \newcommand{\PS}{\mathcal P}
    \newcommand{\Z}{\mathbb{Z}}
    \newcommand{\Q}{\mathbb{Q}}
    \newcommand{\C}{\mathbb{C}}
    \newcommand{\zn}{\zeta_n}
    
    
    
    %
    % The following macro is used to generate the header.
    %
    
    %
    % Convention for citations is authors' initials followed by the year.
    % For example, to cite a paper by Leighton and Maggs you would type
    % \cite{LM89}, and to cite a paper by Strassen you would type \cite{S69}.
    % (To avoid bibliography problems, for now we redefine the \cite command.)
    % Also commands that create a suitable format for the reference list.
    \renewcommand{\cite}[1]{[#1]}
    \def\beginrefs{\begin{list}%
            {[\arabic{equation}]}{\usecounter{equation}
             \setlength{\leftmargin}{2.0truecm}\setlength{\labelsep}{0.4truecm}%
             \setlength{\labelwidth}{1.6truecm}}}
    \def\endrefs{\end{list}}
    \def\bibentry#1{\item[\hbox{[#1]}]}
    
    %Use this command for a figure; it puts a figure in wherever you want it.
    %usage: \fig{NUMBER}{SPACE-IN-INCHES}{CAPTION}
    \newcommand{\fig}[3]{
                \vspace{#2}
                \begin{center}
                Figure \thelecnum.#1:~#3
                \end{center}
        }
    % Use these for theorems, lemmas, proofs, etc.
    \newtheorem{theorem}{Theorem}[lecnum]
    \newtheorem{lemma}[theorem]{Lemma} 
    \newtheorem{proposition}[theorem]{Proposition}
    \newtheorem{claim}[theorem]{Claim}
    \newtheorem{corollary}[theorem]{Corollary}
    \newtheorem{definition}[theorem]{Definition}
    \newtheorem{example}[theorem]{Example}
    
    \newenvironment{proof}{{\bf Proof:}}{\hfill\rule{2mm}{2mm}}
    
    % **** IF YOU WANT TO DEFINE ADDITIONAL MACROS FOR YOURSELF, PUT THEM HERE:
    
    \newcommand\E{\mathbb{E}}
    \newcommand\R{\mathbb{R}}
    

    \usepackage{tikz-cd}
    
    \begin{document}

    \section{Cyclotomic Extensions}

    F field, assume $x^n - 1$ is sep. 

    $F(\zeta_n)/F$, $\zeta_n$ is a primitive $n^{th}$ root of unity. 


    $\{1,\zn, ..., \zn - 1 \}$

    Roots of $x^n - 1$ and $C_n$, $x^n = 1$. 

    These extensions come up $f(x) = x^n - a, a \in F$.

    $\sqrt[n] a, \zn \sqrt[n]{a}, \dots$

    Q: Number of generators of $C_n$?

    $C_n = <b>$.

    $|b| = n$

    $|b^j| = \frac{n}{(j,n)}$
  
    Number of generators of $|(\Z/n\Z)^\times| = \phi(n)$.

    $\phi(p) = p - 1$

    $\phi(p^r) = p^r - p^{r-1}(p-1)$

    $\{1,2, \dots, p^r\}$

    \begin{theorem}
        $a.b \in \Z$ are coprime, $\Z/ab\Z \rightarrow \Z/a\Z \times \Z/b\Z$. 

        Ex: $\Z/6\Z \rightarrow \Z/2\Z \times \Z/3\Z$

        $a + 6z \rightarrow \alpha + 2\Z, \alpha + 3\Z$.

        Ex: $\Z/4\Z \rightarrow \Z/2\Z \times \Z/2\Z$

        $a + 4z \rightarrow \alpha + 2\Z, \alpha + 2\Z$.
    \end{theorem}

    By induction, suppose that $n = p_1^{r_1}\dots p_k^{r_k}$.

    $p_1^{r_1}\dots$ distinct primes. 

    $\Z/n\Z = (\Z/p^{r_1}\Z)^{\times} \times \dots \times (\Z/p^{r_k}\Z)^{\times}$

    $\phi(n) = \phi(p_1^{r_1})\dots\phi(p_k^{r_k}) = n\underset{p \vert n}\Pi(1 - \frac1p)$


    \section{Study $Gal(F(\zn)/F)$}

    Lemma $\sigma \in Gal(F(\zn)/F)$

    and let $\zeta \in \mu_n(F(\zn)) = \{\zn^k \vert 1 \leq k \leq n\}$ then $\exists a = a_\sigma, (a,n) = 1$ s.t. $\sigma(\zeta) = \zeta^a$

    Rem:

    $Gal(F(\zeta_n)/F) \rightarrow Perm(\{1,\zn, ..., \zn - 1 \})$

    Pf: $\zn$ is primitive, $\zn = 1$, $\zn^j \neq 1$, $1 \leq j \leq n$.

    $\sigma(\zn)$ must be a generator. 

    $\implies \sigma(\zn) = \zn^a$, $(a,n) = 1$ for some $a$.

    $\sigma(\zeta) = \sigma(\zeta^k_n) = \sigma(\zeta_m)^k = (\zeta_n^a)^k = (\zeta_n^k)^a$

    Thm: $Gal(F(\zn)/F) \rightarrow (\Z/n\Z)^\times$

    $\sigma \rightarrow a_\sigma$ an injective homo. 


    $\sigma \tau = a_\sigma a_\tau$

    $\zeta^{a_{\tau\sigma}} = (\tau\sigma)(\zeta) = \sigma(\tau(\zeta)) = \sigma(\zeta^{a\tau}) = (\sigma(\zeta))^{a\tau} = (\zeta^{a\sigma})^{a\tau} = \zeta^{a_\sigma, a_\tau}$ 

    $\sigma \in$ kernel : $\sigma(\zeta_n) = \zeta^1_n$


    $F(\zn)$ \ \ \  \ $\sigma_{\vert F} = id$

    generated by $F, \zn \implies \sigma = id$. 

    Cor: $Gal(F(\zn)/F)$ abelian. 

    Next: $Gal(\Q(\zn)/\Q) \cong (\Z/n\Z)^\times$.
    
    \section{Cyclotomic Polynomials}

    $\Q, x^n - 1$, $\zeta = \zn$ primitive.

    Def $\Phi_n(x) = \underset{1 \leq i \leq n \\ (i,n) = 1}\Pi(x - \zeta^i)$

    $\Phi_2(x) = x -(-1) = x+1$

    $\Phi (x) = (x - \zeta)(x - \zeta^2), \zeta^3 = 1 = x^2 - (\zeta + \zeta^2)x + 1$

    $= x^2 + x+1$

    $\Phi_4(x) = x = (x+i)(x-i) = x^2 + 1$ 



\end{document}


