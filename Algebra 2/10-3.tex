\documentclass[twoside]{article}
    \setlength{\oddsidemargin}{0.25 in}
    \setlength{\evensidemargin}{-0.25 in}
    \setlength{\topmargin}{-0.6 in}
    \setlength{\textwidth}{6.5 in}
    \setlength{\textheight}{8.5 in}
    \setlength{\headsep}{0.75 in}
    \setlength{\parindent}{0 in}
    \setlength{\parskip}{0.1 in}
    
    %
    % ADD PACKAGES here:
    %
    
    \usepackage{amsmath,amsfonts,amssymb,graphicx,mathtools,flexisym}
    
    %
    % The following commands set up the lecnum (lecture number)
    % counter and make various numbering schemes work relative
    % to the lecture number.
    %
    \newcounter{lecnum}
    \renewcommand{\thepage}{\thelecnum-\arabic{page}}
    \renewcommand{\thesection}{\thelecnum.\arabic{section}}
    \renewcommand{\theequation}{\thelecnum.\arabic{equation}}
    \renewcommand{\thefigure}{\thelecnum.\arabic{figure}}
    \renewcommand{\thetable}{\thelecnum.\arabic{table}}
    \newcommand{\N}{\mathbb{N}}
    \newcommand{\PS}{\mathcal P}
    \newcommand{\Z}{\mathbb{Z}}
    \newcommand{\Q}{\mathbb{Q}}
    \newcommand{\F}{\mathbb{F}}
    \newcommand{\C}{\mathbb{C}}
    \newcommand{\zn}{\zeta_n}
    \newcommand{\z}{\zeta}
    
    
    
    %
    % The following macro is used to generate the header.
    %
    
    %
    % Convention for citations is authors' initials followed by the year.
    % For example, to cite a paper by Leighton and Maggs you would type
    % \cite{LM89}, and to cite a paper by Strassen you would type \cite{S69}.
    % (To avoid bibliography problems, for now we redefine the \cite command.)
    % Also commands that create a suitable format for the reference list.
    \renewcommand{\cite}[1]{[#1]}
    \def\beginrefs{\begin{list}%
            {[\arabic{equation}]}{\usecounter{equation}
             \setlength{\leftmargin}{2.0truecm}\setlength{\labelsep}{0.4truecm}%
             \setlength{\labelwidth}{1.6truecm}}}
    \def\endrefs{\end{list}}
    \def\bibentry#1{\item[\hbox{[#1]}]}
    
    %Use this command for a figure; it puts a figure in wherever you want it.
    %usage: \fig{NUMBER}{SPACE-IN-INCHES}{CAPTION}
    \newcommand{\fig}[3]{
                \vspace{#2}
                \begin{center}
                Figure \thelecnum.#1:~#3
                \end{center}
        }
    % Use these for theorems, lemmas, proofs, etc.
    \newtheorem{theorem}{Theorem}[lecnum]
    \newtheorem{lemma}[theorem]{Lemma} 
    \newtheorem{proposition}[theorem]{Proposition}
    \newtheorem{claim}[theorem]{Claim}
    \newtheorem{corollary}[theorem]{Corollary}
    \newtheorem{definition}[theorem]{Definition}
    \newtheorem{example}[theorem]{Example}
    
    \newenvironment{proof}{{\bf Proof:}}{\hfill\rule{2mm}{2mm}}
    
    % **** IF YOU WANT TO DEFINE ADDITIONAL MACROS FOR YOURSELF, PUT THEM HERE:
    
    \newcommand\E{\mathbb{E}}
    \newcommand\R{\mathbb{R}}
    

    \usepackage{tikz-cd}
    
\begin{document}

Goal: $Gal(\Q(\zn)/\Q) = \Z / n\Z^\times$

$\zeta = \zn$, primitive $n^{th}$ root of 1. 

Def $\Phi(x) = \underset{1 \leq i \leq n}\Pi(x - \zeta^i)$
 n-th cyclotomic polynomial. 

Examples: $\Phi_1(x) = x - 1$

Examples: $\Phi_2(x) =x + 1$

Examples: $\Phi_3(x) = x^2 + x + 11$

Examples: $\Phi_4(x) = x^2 + 1$

Examples: $\Phi_5(x) = x^4 + x^3 +x^2 +x + 1$

Observe $x^n - 1 = \underset{d \vert n }\Pi\Phi_d(x) = \underset{1 \leq i \leq n}\Pi(x - \zeta^i)$

In $C_n = <\z>$, we have elements of order $d \vert n$

                        $C_n$ 
            $C_{d_1}$                $C_{d_2}$
    $C_{d_4}$          $C_{d_3}$


Claim: $\Phi_n(x) \in \Z[x] \forall n \in \N$

\begin{proof}
    Induction on $n$. 

    $n = 1 = x-1 \in \Z[x]$

    Assume    $\Phi_m(x) \in \Z[x] \  \forall m < n $

    $x^n-1 = \Phi_n(x) = \underset{d \vert n \ \ d \neq n}\Pi\Phi_d(x) \in \Z[x]$.

    By Gauss' Lemma $\Phi_n(x) \in \Z[x]$. 
\end{proof}

\begin{theorem}
    $\Phi_n(x)$ is irreducible over $\Q$ for $\forall n \in \N$.
\end{theorem}

\begin{proof}
    Let $f(x)$ be the minimal polynomial of $\zeta = \zeta_n$. 

    $\phi_n(x) = \underset{1 \leq i \leq n \\ \ (i,n) = 1}\Pi(x - \zeta^i)$

    $\implies f(x) \vert \Phi_n(x)$.

    To show that $f(x) = \Phi_n(x)$, we show that $\z^i$ is a root of $f$, $\forall i,\ (i,n) = 1$. 

    Enough to show that $\z^p$ is a root of $f$ for all primes $p$, $(p,n) = 1$. 

    Enoguh to show that $f(\eta) = 0 $ then $f(\eta^p) = 0$ for $(p,n) = 1$ . 

    $i = p_1p_2\dots p_r$. $(p_j, n) = 1$

    $\z, \z^{p_1}, \z^{p_1p_2}, {\z^{\overset{=i}{p_1\dots p_r}}} = \z^i$

    Claim: If $f(\eta) = 0, (p,n) = 1$ then $f(\eta^p) = 0$

    \begin{proof}
        Suppose for a contradiction, $f(\eta) = 0$, $f(\eta^p) = 0$. 

        Write $\Phi_n(x) = f(x)g(x)$. 

        So $g(\eta^p) = 0$

        $\implies \eta$ is a root of $g(x^p)$

        By Gauss' Lemma, $f(x), g(x) \in \Z[x]$

        $f(x) \vert g(x^p)$

        Reduce modulo $p$. $\overline f(x)$, $\overline g(x^p) \in \F_p$

        $\overline {f(x)} \vert \overline {g(x^p)} = \overline {g(x)^p} $

        $\implies \overline {f(x)}, \overline {g(x)}$ have common roots. 

        $f(x)g(x) = \Phi_n(x) \vert x^n - 1$

        $\overline {f(x)g(x)} = \overline{\Phi_n(x)} \vert \overline{x^n - 1}$

        $\overline {f(x)g(x)}$ has multiple roots. 

        But $\overline{x^n - 1}$ cannot have multiple roots. 

        Recall: $h(x) \in \F_p[x]$ has  multiple roots iff $gcd(h,h') \neq 1$.

        But $(\overline{x^n - 1})' = \overline{nx^{n-1}}$, only $\overline 0$ is a root. 

        $\implies $ claim holds. 

    \end{proof}
\end{proof}

Cor: $Gal(\Q(\zeta_n)/\Q) \cong (\Z/n\Z)^\times$

\begin{proof}
    $[\Q[\zeta_n] : \Q] = deg \Phi_n(x) = \phi(n)$

    $|(\Z/n\Z)| = \phi(n)$

    Gal$(\Q(\z_n)/\Q) \hookrightarrow \Z/n\Z^x$.

\end{proof}

\begin{theorem}
    Any finite Galois extension $E/\Q$ with abelian Galois gp is ismorphic to a subfield of $\Q(\z_n)$ for some $n$. 

    $E \hookrightarrow \Q(\z_n)$
   (ableian) $\Q$                  $\Q$


\end{theorem}
\end{document}


