\documentclass[twoside]{article}
    \setlength{\oddsidemargin}{0.25 in}
    \setlength{\evensidemargin}{-0.25 in}
    \setlength{\topmargin}{-0.6 in}
    \setlength{\textwidth}{6.5 in}
    \setlength{\textheight}{8.5 in}
    \setlength{\headsep}{0.75 in}
    \setlength{\parindent}{0 in}
    \setlength{\parskip}{0.1 in}
    
    %
    % ADD PACKAGES here:
    %
    
    \usepackage{amsmath,amsfonts,amssymb,graphicx,mathtools,flexisym}
    
    %
    % The following commands set up the lecnum (lecture number)
    % counter and make various numbering schemes work relative
    % to the lecture number.
    %
    \newcounter{lecnum}
    \renewcommand{\thepage}{\thelecnum-\arabic{page}}
    \renewcommand{\thesection}{\thelecnum.\arabic{section}}
    \renewcommand{\theequation}{\thelecnum.\arabic{equation}}
    \renewcommand{\thefigure}{\thelecnum.\arabic{figure}}
    \renewcommand{\thetable}{\thelecnum.\arabic{table}}
    \newcommand{\N}{\mathbb{N}}
    \newcommand{\PS}{\mathcal P}
    \newcommand{\Z}{\mathbb{Z}}
    \newcommand{\Q}{\mathbb{Q}}
    \newcommand{\F}{\mathbb{F}}
    \newcommand{\C}{\mathbb{C}}
    \newcommand{\zn}{\zeta_n}
    \newcommand{\z}{\zeta}
    \newcommand{\al}{\alpha}
    
    
    
    %
    % The following macro is used to generate the header.
    %
    
    %
    % Convention for citations is authors' initials followed by the year.
    % For example, to cite a paper by Leighton and Maggs you would type
    % \cite{LM89}, and to cite a paper by Strassen you would type \cite{S69}.
    % (To avoid bibliography problems, for now we redefine the \cite command.)
    % Also commands that create a suitable format for the reference list.
    \renewcommand{\cite}[1]{[#1]}
    \def\beginrefs{\begin{list}%
            {[\arabic{equation}]}{\usecounter{equation}
             \setlength{\leftmargin}{2.0truecm}\setlength{\labelsep}{0.4truecm}%
             \setlength{\labelwidth}{1.6truecm}}}
    \def\endrefs{\end{list}}
    \def\bibentry#1{\item[\hbox{[#1]}]}
    
    %Use this command for a figure; it puts a figure in wherever you want it.
    %usage: \fig{NUMBER}{SPACE-IN-INCHES}{CAPTION}
    \newcommand{\fig}[3]{
                \vspace{#2}
                \begin{center}
                Figure \thelecnum.#1:~#3
                \end{center}
        }
    % Use these for theorems, lemmas, proofs, etc.
    \newtheorem{theorem}{Theorem}[lecnum]
    \newtheorem{lemma}[theorem]{Lemma} 
    \newtheorem{proposition}[theorem]{Proposition}
    \newtheorem{claim}[theorem]{Claim}
    \newtheorem{corollary}[theorem]{Corollary}
    \newtheorem{definition}[theorem]{Definition}
    \newtheorem{example}[theorem]{Example}
    
    \newenvironment{proof}{{\bf Proof:}}{\hfill\rule{2mm}{2mm}}
    
    % **** IF YOU WANT TO DEFINE ADDITIONAL MACROS FOR YOURSELF, PUT THEM HERE:
    
    \newcommand\E{\mathbb{E}}
    \newcommand\R{\mathbb{R}}
    

    \usepackage{tikz-cd}
    \usepackage{amssymb}
    
\begin{document}
\begin{theorem}
    The regular n-gon is constructible iff $n = 2^rp_1p_2\dots p_k$, $p_i$ Fermat primes. 
\end{theorem}

Recall: A number $\al$ is constructible iff $\Q(\alpha) \supset E_j \subset \dots \supset E_2 \supset E_1 \subset E = \Q$
$E_{i+1}/E_i$ is quadratic. $E_{i+1} = E_i(\sqrt{a_i})$

A prime $p$ is Fermat iff $p = 2^{2^a} + 1$, 

$3,5,17,257, 65537, \dots$


\begin{proof}
    Both sides are equivalent to $\phi(n) =2^i$. 

    $\circ$ $\phi(n) = p_1^{r_1-1}(p_1 -1) \dots p_k^{r_k-1}(p_k - 1)$; 

    $n = p_1^{r_1}\dots p_k^{r_k}$

    Power of $2 \iff$ all $p_i's \neq 2$ appear with multiplicity 1 and are Fermat.

    $\circ\circ$ Regular $n-gon$ constructible $\iff$ $Re(\z_n) = cos(\frac{2\pi}{n})$ is constructible.constructible

    $cos(\frac{2\pi}{n})$ iff angle $\frac{2\pi}{n}$ constructible. 

    $\iff \Q(\cos\frac{2\pi}{n}/\Q)$ is a tower of quadratic extensions. 

    $\alpha = \frac12 (\zeta_n + \zeta_n^{-1}) = \frac12 (e^{2i\pi}{n} + e^{-2i\pi}{n})$

    Tower $\Q(\z_n) -(2)- \Q(\alpha) ---- \Q$

    $f(x) = (X - \zeta_n)(X - \zeta_n^{-1})$

    $=x^2 - (\z_n + \z_n^{-1})x + 1$

    $= x^2 - 2\alpha x + 1$


    $\Q(\al)/\Q$ is Galois because subgroups of abelian groups are normal. 

    Lemma: Let $E/F$ is Galois with abelian Galois group. Then $E$ is a tower of quadratic extensions of $F \iff [E:F] = $ power of 2
    
    $\implies$ clear. 

    $\impliedby$ Induction on $[E:F]$ 

    $[E:F] = 1$ true.

    $[E:F] > 1$. Choose $g \in Gal(E/F)$ of order 2. 

    $<g> \unlhd \ Gal(E/F)$

    $\implies E^{<g>}/F$ Galois, then use induction.
\end{proof}


\subsection{Radical Extensions}

\begin{definition}
    A finite extension $L/F$ is solvable by radicals if there is a tower of extensions $F = F_0 \hookrightarrow F_2 \hookrightarrow \dots \hookrightarrow F_m$

    such that
    
    \begin{itemize}
        \item      $\forall i \exists n, a\in F_i $ s.t. $F_{i+1} = F_i(\sqrt[n]{a})$
        \item $L \subseteq F_m$
    \end{itemize}
\end{definition}

\begin{example}
    $F \subset F(\sqrt[n]a), a \subset F$

    Let $\zeta_n$ be a primitive $n^{th}$ root of 1.

    Assume $char(F), \not \vert n \implies x^n - a$ has no multiple roots. 

               $E = F(\sqrt[n]{a},\zeta_n)$
        Galois                              Galois
        $F(\sqrt[n]a)$                 $F(\sqrt(\zeta_n))$
                                        Galois
                            $F$

        $Gal(F(\z_n)/F) \hookrightarrow (\Z/n\Z)^{\times}$

        $Gal(F(\sqrt[n]{a},\zeta_n)/F(\z_n)) \hookrightarrow \Z/n\Z $ (addition)

        $Gal(F(\sqrt[n]{a},\zeta_n)/F(\z_n)) = $
\end{example}

\end{document}



