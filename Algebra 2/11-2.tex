\documentclass[twoside]{article}
    \setlength{\oddsidemargin}{0.25 in}
    \setlength{\evensidemargin}{-0.25 in}
    \setlength{\topmargin}{-0.6 in}
    \setlength{\textwidth}{6.5 in}
    \setlength{\textheight}{8.5 in}
    \setlength{\headsep}{0.75 in}
    \setlength{\parindent}{0 in}
    \setlength{\parskip}{0.1 in}
    
    %
    % ADD PACKAGES here:
    %
    
    \usepackage{amsmath,amsfonts,amssymb,graphicx,mathtools,flexisym}
    
    %
    % The following commands set up the lecnum (lecture number)
    % counter and make various numbering schemes work relative
    % to the lecture number.
    %
    \newcounter{lecnum}
    \renewcommand{\thepage}{\thelecnum-\arabic{page}}
    \renewcommand{\thesection}{\thelecnum.\arabic{section}}
    \renewcommand{\theequation}{\thelecnum.\arabic{equation}}
    \renewcommand{\thefigure}{\thelecnum.\arabic{figure}}
    \renewcommand{\thetable}{\thelecnum.\arabic{table}}
    \newcommand{\N}{\mathbb{N}}
    \newcommand{\PS}{\mathcal P}
    \newcommand{\Z}{\mathbb{Z}}
    \newcommand{\Q}{\mathbb{Q}}
    \newcommand{\F}{\mathbb{F}}
    \newcommand{\C}{\mathbb{C}}
    \newcommand{\zn}{\zeta_n}
    \newcommand{\z}{\zeta}
    \newcommand{\al}{\alpha}
    \newcommand{\ra}{\rightarrow}
    \newcommand{\hra}{\hookrightarrow}
    
    
    
    %
    % The following macro is used to generate the header.
    %
    
    %
    % Convention for citations is authors' initials followed by the year.
    % For example, to cite a paper by Leighton and Maggs you would type
    % \cite{LM89}, and to cite a paper by Strassen you would type \cite{S69}.
    % (To avoid bibliography problems, for now we redefine the \cite command.)
    % Also commands that create a suitable format for the reference list.
    \renewcommand{\cite}[1]{[#1]}
    \def\beginrefs{\begin{list}%
            {[\arabic{equation}]}{\usecounter{equation}
             \setlength{\leftmargin}{2.0truecm}\setlength{\labelsep}{0.4truecm}%
             \setlength{\labelwidth}{1.6truecm}}}
    \def\endrefs{\end{list}}
    \def\bibentry#1{\item[\hbox{[#1]}]}
    
    %Use this command for a figure; it puts a figure in wherever you want it.
    %usage: \fig{NUMBER}{SPACE-IN-INCHES}{CAPTION}
    \newcommand{\fig}[3]{
                \vspace{#2}
                \begin{center}
                Figure \thelecnum.#1:~#3
                \end{center}
        }
    % Use these for theorems, lemmas, proofs, etc.
    \newtheorem{theorem}{Theorem}[lecnum]
    \newtheorem{lemma}[theorem]{Lemma} 
    \newtheorem{proposition}[theorem]{Proposition}
    \newtheorem{claim}[theorem]{Claim}
    \newtheorem{corollary}[theorem]{Corollary}
    \newtheorem{definition}[theorem]{Definition}
    \newtheorem{example}[theorem]{Example}
    
    \newenvironment{proof}{{\bf Proof:}}{\hfill\rule{2mm}{2mm}}
    
    % **** IF YOU WANT TO DEFINE ADDITIONAL MACROS FOR YOURSELF, PUT THEM HERE:
    
    \newcommand\E{\mathbb{E}}
    \newcommand\R{\mathbb{R}}
    

    \usepackage{tikz-cd}
    \usepackage{amssymb}
    
\begin{document}

Def: A finite extension $L/F$ is solvable by radicals if $L \subset $ iterated radical extension. 


$F  = F_0 \subset F_1 \subset \dots \subset F_m$

$F_{i+1} = F_i(\sqrt[n_i]{a_i})$

$E = F(\sqrt[n](a), \z_n)$ splitting of $x^n -a$

                $F(\sqrt[n]{a}, \z_n)$
$F(\sqrt[n]{a})$                      $F(\z_n)$
                        $F$ 

Know $Gal(F(\z_n)/F) \hookrightarrow (\Z/nZ)^\times$


$Gal(\sqrt[n]{a}, F(\z_n)/F) \hookrightarrow (\Z/nZ)$
because $\sigma(\sqrt[n]a) = \z_n^i \sqrt[n]a$


Some $1 \leq i \leq n$. 

$\sigma \mapsto i_\sigma$
$\tau \mapsto i_\tau$

$\sigma\tau \mapsto i_{\sigma\tau}$

$(\sigma\tau)(\al) = \sigma(\tau(\alpha)) = \sigma(\z_n^{i\tau}\cdot\alpha) = \sigma(\z^{i\tau})\times\sigma(\alpha) = \zeta^{i_\tau}\cdot \zeta^{i_\sigma}\alpha = i_\tau + i_\sigma$

$\Z/n\Z \supset Gal(F(\alpha, \z)/F(\Z)) \hra Gal(F(\alpha,\zeta)/F) \twoheadrightarrow Gal(F(\z_n)/F) \subset (\Z/n\Z)^\times$

\begin{definition}
    A finite group $G$ is solvable if $\exists$ subgroups of $G_i$ such that ${1} = G_m \subseteq \dots \subseteq G_1 \subseteq G_0 = G$
    s.t. $G_{i+1} \triangleleft G_i$ and $G_i/G_{i+i}$ and $G_i/G_{i+1}$. 
\end{definition}


\begin{example}
    Abelian Groups

    $D_{2n}$ dihedral groups. 
    
    $\Z/n\Z \subset D_{2n}$, $D_2n/\Z/n\Z \cong \Z/2\Z$        

    ${e} \subset \Z/n\Z \subset D_{2n}$, $D_{2n}/\Z/n\Z \cong \Z/2\Z$

    $D_{10} = <\sigma, \tau>$

    $<\sigma> \cong \Z/5\Z$

    $<\tau> \cong \Z/2\Z$

    $GL_n(F) = $ invertible matrices.

    $G = \{\text{upper triangular matrices}\} <  GL_n(F)$

\end{example}

Fact: $S_n$ is solvable $\iff n < 5$. (Need to prove)

\begin{theorem}
    Let $N \triangleleft G$. Then $G$ is solvable $\iff G/N$ are solvable. 
\end{theorem}
\begin{proof}
    ($\implies$)

    $G = G_0 \supseteq G_1 \supseteq G_1 \supseteq \dots \supseteq G_m = \{1\}$, $G_i/G_{i+1}$ normal. 

    Need to find $G/N = H_0 \supseteq H_1 \supseteq \dots \supseteq H_? = 1$

    Let $H_i$ image of $G_i$ in $G/N$. 

    \newpage
        $G \ra G/N$
        
        $\vee$ \ \ \ \  \ \  $\vee$

        $G_i \ra G_iN/N = H_i$ 

        $H_{i+1} \triangleleft H_{i}$

        $G_{i+1}N/N \triangleleft G_iN/N$

        $g_inN$

    $G_i \twoheadrightarrow \ \ \ \  \ \ \ H$ 

    $\downarrow$ \ \ \ \ \ \ \ \     $\downarrow$ 
    
    $G_i/G_{i+1}$          $\rightarrow[F]$          $H_i/H$

    $\implies H_i/H_{i+1}$ Abelian (quotient of Abelian)

\end{proof}


\end{document}



