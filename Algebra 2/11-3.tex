\documentclass[twoside]{article}
    \setlength{\oddsidemargin}{0.25 in}
    \setlength{\evensidemargin}{-0.25 in}
    \setlength{\topmargin}{-0.6 in}
    \setlength{\textwidth}{6.5 in}
    \setlength{\textheight}{8.5 in}
    \setlength{\headsep}{0.75 in}
    \setlength{\parindent}{0 in}
    \setlength{\parskip}{0.1 in}
    
    %
    % ADD PACKAGES here:
    %
    
    \usepackage{amsmath,amsfonts,amssymb,graphicx,mathtools,flexisym}
    
    %
    % The following commands set up the lecnum (lecture number)
    % counter and make various numbering schemes work relative
    % to the lecture number.
    %
    \newcounter{lecnum}
    \renewcommand{\thepage}{\thelecnum-\arabic{page}}
    \renewcommand{\thesection}{\thelecnum.\arabic{section}}
    \renewcommand{\theequation}{\thelecnum.\arabic{equation}}
    \renewcommand{\thefigure}{\thelecnum.\arabic{figure}}
    \renewcommand{\thetable}{\thelecnum.\arabic{table}}
    \newcommand{\N}{\mathbb{N}}
    \newcommand{\PS}{\mathcal P}
    \newcommand{\Z}{\mathbb{Z}}
    \newcommand{\Q}{\mathbb{Q}}
    \newcommand{\F}{\mathbb{F}}
    \newcommand{\C}{\mathbb{C}}
    \newcommand{\zn}{\zeta_n}
    \newcommand{\z}{\zeta}
    \newcommand{\al}{\alpha}
    \renewcommand{\a}{\alpha}
    \newcommand{\ra}{\rightarrow}
    \newcommand{\hra}{\hookrightarrow}
    \newcommand{\ep}{\varepsilon}   
    \newcommand{\trr}{\triangleright}   
    \newcommand{\trl}{\triangleleft}   
    \newcommand{\thra}{\twoheadrightarrow}   
    
    
    %
    % The following macro is used to generate the header.
    %
    
    %
    % Convention for citations is authors' initials followed by the year.
    % For example, to cite a paper by Leighton and Maggs you would type
    % \cite{LM89}, and to cite a paper by Strassen you would type \cite{S69}.
    % (To avoid bibliography problems, for now we redefine the \cite command.)
    % Also commands that create a suitable format for the reference list.
    \renewcommand{\cite}[1]{[#1]}
    \def\beginrefs{\begin{list}%
            {[\arabic{equation}]}{\usecounter{equation}
             \setlength{\leftmargin}{2.0truecm}\setlength{\labelsep}{0.4truecm}%
             \setlength{\labelwidth}{1.6truecm}}}
    \def\endrefs{\end{list}}
    \def\bibentry#1{\item[\hbox{[#1]}]}
    
    %Use this command for a figure; it puts a figure in wherever you want it.
    %usage: \fig{NUMBER}{SPACE-IN-INCHES}{CAPTION}
    \newcommand{\fig}[3]{
                \vspace{#2}
                \begin{center}
                Figure \thelecnum.#1:~#3
                \end{center}
        }
    % Use these for theorems, lemmas, proofs, etc.
    \newtheorem{theorem}{Theorem}[lecnum]
    \newtheorem{lemma}[theorem]{Lemma} 
    \newtheorem{proposition}[theorem]{Proposition}
    \newtheorem{claim}[theorem]{Claim}
    \newtheorem{corollary}[theorem]{Corollary}
    \newtheorem{definition}[theorem]{Definition}
    \newtheorem{example}[theorem]{Example}
    
    \newenvironment{proof}{{\bf Proof:}}{\hfill\rule{2mm}{2mm}}
    
    % **** IF YOU WANT TO DEFINE ADDITIONAL MACROS FOR YOURSELF, PUT THEM HERE:
    
    \newcommand\E{\mathbb{E}}
    \newcommand\R{\mathbb{R}}
    

    \usepackage{tikz-cd}
    \usepackage{amssymb}
    
\begin{document}

\subsection{Preparation for Galois' Solvability Theorem}

Def: $G$ is called solvable if there exists a sequence of subgroups $G = G_0 \triangleright G_1 \triangleright \dots \trr G_m  = \{1\}$.

s.t. $G_i/G_{i+1}$ is abelian. 

\begin{theorem}
    Let $N \trr G$. $G$ is solvable $\iff N, G/N$ is solvable.  
\end{theorem}

\begin{proof}
    $\implies$ 
    
    $G/N \checkmark$

    $N_i = G_i \cap N$. 

    $N = N_0 \supset N_1 \supset \dots \supset N_m = 1$.

    $N_{i+1} \trl N_i$? Yes, because $N \cap G_{i+1} \trl G_i$

    $N_i/N_{i+1}$ Abelian? $N \cap G_i \hra G_i \twoheadrightarrow G_i/G_{i+1}$ 

    $f: N \cap G_i \rightarrow G_i/G_{i+1}$. composition of above. 

    Kernel of $f$? $n \rightarrow nG_{i+1}$, $n \in \ker f \iff n \in G_{i+1} \cap N$. 


    $N_i/N_{i+1} = N\cap G_i/N\cap G_{i+1} \cong \text{im} f < G_i/G_{i+1}$. 

    $\impliedby$

    Construct a series for $G$. 

    $N = N_0 \trr N_1 \trr \dots \trr N_m = \{1\}$

    $G/N = H_0 \trr H_1 \trr \dots \trr H_m = \{1\}$

    $\{1\}= N_m \trl \dots \trl N_0 = N - G_n < G_{n-1} \dots G < G $

    \ \ \ \ \ \ \ \ \ \ \ \ \ \ \ \ \ \ \ \ \ \ \ \ \ \ \ \    $\{1\} \trl H_n \dots \trl H_0 = G/N$

    $\phi : G \ra G/N$
    
    \ \ \ \ $\vee$     \ \ \ \ $\vee$


    $\phi^{-1}(H_1) \rightarrow H_i$


So $G_{i+1} \trl G_i$ because $H_{i+1} \trl H_i$. 

$G_i/G_{i+1} \cong {G_i/N}/{G_{i+1}/N} = H_i/H_{i+1}$

By 3rd homo. them. 

$\therefore G_i/G_{i+1}$ abelian. 

\end{proof}

\subsection{Cyclic Extensions}. 

\begin{theorem}
    (Dedekind) Let $F$ be a field, and $G$ a group. Then every finite set $\{\chi_1, \dots, \chi_m\}$ of homomorphisms 

    $G_i : G \rightarrow F^{\times}$ is linearly independent over $F$
\end{theorem}

Remark: $X$ set, $F$ field. 

$Func (X,F) = \{f : X \rightarrow F\}$ is a vector space over f. 

$(f_1 + f_2)(x) = f_1(x) + f_2(x)$. 

$(\alpha \cdot f)(x) = \a f(x)$.

$\chi_1, \dots, \chi_m \in Func(G,F)$ are linearly independent.


\begin{theorem}
    $\sum_ia_i\chi_i = 0 \implies a_1 = \dots = a_m = 0$.
\end{theorem}

\begin{proof}
    $m = 1 \checkmark$.

    $a \chi = 0 \implies a = 0$. 

    Assume $m -1$. 

    $a_1 \chi_1 + a_m \chi_m = 0$. $\star$

    $a_i \in F$ need to show all zero.

    $\chi_1 \neq \chi_2 \implies \exists g \in G$

    $\chi_1(g) \neq \chi_2(g)$

    $\forall x \in G: a_1\chi_1 + \dots + a_m\chi_m = 0$

    also for $gx: a_1\chi_1(gx) + \dots + a_m\chi_m(gx) = 0$

    (*) $a_1\chi_1(x)\chi_1(g) + \dots + a_m\chi_m(x)\chi_m(g) = 0$.

    (**) $a_1\chi_1(x)\chi_1(g) + \dots + a_m\chi_1(x)\chi_m(g) = 0$. By mult above with $\chi_1(x)$

    $(*) - (**) = \sum_{j=2}^m a_j'\chi_j(x) = 0$. $\forall x \in G$. 


    $a_j' = a_j(\chi_j(g) - \chi_1(g))$. 

    By induction, $a_j' = 0$. 

    In particular, $a_2' = 0$.

    $0 = a_2' = a_2 (x_2(g) - x_1(g)) \neq 0$

    $\implies a_2 = 0$

    So in $\star$, there are $m-1$ terms, by induction $a_1 = a_3 = \dots = a_m = 0$

\end{proof}

\newpage

\subsection{Back to Cyclic Extensions}

$F = F_0 \subset F_1 \subset \dots F_m$.

$F_{i+1} = F_i(\sqrt[n_i]{a_i})$

\begin{theorem}
    Let F be a field containing a primitive $n^{th}$ root of 1. Let $E = F[\a], \a^n = a \in F$ and no smaller power of $\a \in F$. Then $E/F$ is Galois ext with $Gal(E/F) \cong \Z/n\Z$.

    Conversely, if $E/F$ is cyclic Galois Ext of degree $n$, then $\exists \alpha \in E$ s.t. $E = F[\a], \a^n \in F$. 
\end{theorem}

\begin{proof}
    $(\implies)$


    $\alpha, \zeta\alpha, \z^2\a, \dots, \z^n\a$ are the roots of $x^n - a \in F[x]$. 

    $Gal(F[\alpha]/F) \rightarrow \Z/n\Z$

    $\sigma \rightarrow i\sigma$, $\sigma(\alpha) = \z^{i\sigma}\alpha$

    $\impliedby$

    enough to find $\alpha \in E^\times$ s.t. $\sigma(\alpha) = \zeta^{-1}\alpha$

\end{proof}

\end{document}



