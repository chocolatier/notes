\documentclass[twoside]{article}
    \setlength{\oddsidemargin}{0.25 in}
    \setlength{\evensidemargin}{-0.25 in}
    \setlength{\topmargin}{-0.6 in}
    \setlength{\textwidth}{6.5 in}
    \setlength{\textheight}{8.5 in}
    \setlength{\headsep}{0.75 in}
    \setlength{\parindent}{0 in}
    \setlength{\parskip}{0.1 in}
    
    %
    % ADD PACKAGES here:
    %
    
    \usepackage{amsmath,amsfonts,amssymb,graphicx,mathtools,flexisym}
    
    %
    % The following commands set up the lecnum (lecture number)
    % counter and make various numbering schemes work relative
    % to the lecture number.
    %
    \newcounter{lecnum}
    \renewcommand{\thepage}{\thelecnum-\arabic{page}}
    \renewcommand{\thesection}{\thelecnum.\arabic{section}}
    \renewcommand{\theequation}{\thelecnum.\arabic{equation}}
    \renewcommand{\thefigure}{\thelecnum.\arabic{figure}}
    \renewcommand{\thetable}{\thelecnum.\arabic{table}}
    \newcommand{\N}{\mathbb{N}}
    \newcommand{\PS}{\mathcal P}
    \newcommand{\Z}{\mathbb{Z}}
    \newcommand{\Q}{\mathbb{Q}}
    
    
    %
    % The following macro is used to generate the header.
    %
    
    %
    % Convention for citations is authors' initials followed by the year.
    % For example, to cite a paper by Leighton and Maggs you would type
    % \cite{LM89}, and to cite a paper by Strassen you would type \cite{S69}.
    % (To avoid bibliography problems, for now we redefine the \cite command.)
    % Also commands that create a suitable format for the reference list.
    \renewcommand{\cite}[1]{[#1]}
    \def\beginrefs{\begin{list}%
            {[\arabic{equation}]}{\usecounter{equation}
             \setlength{\leftmargin}{2.0truecm}\setlength{\labelsep}{0.4truecm}%
             \setlength{\labelwidth}{1.6truecm}}}
    \def\endrefs{\end{list}}
    \def\bibentry#1{\item[\hbox{[#1]}]}
    
    %Use this command for a figure; it puts a figure in wherever you want it.
    %usage: \fig{NUMBER}{SPACE-IN-INCHES}{CAPTION}
    \newcommand{\fig}[3]{
                \vspace{#2}
                \begin{center}
                Figure \thelecnum.#1:~#3
                \end{center}
        }
    % Use these for theorems, lemmas, proofs, etc.
    \newtheorem{theorem}{Theorem}[lecnum]
    \newtheorem{lemma}[theorem]{Lemma}
    \newtheorem{proposition}[theorem]{Proposition}
    \newtheorem{claim}[theorem]{Claim}
    \newtheorem{corollary}[theorem]{Corollary}
    \newtheorem{definition}[theorem]{Definition}
    \newtheorem{example}[theorem]{Example}
    
    \newenvironment{proof}{{\bf Proof:}}{\hfill\rule{2mm}{2mm}}
    
    % **** IF YOU WANT TO DEFINE ADDITIONAL MACROS FOR YOURSELF, PUT THEM HERE:
    
    \newcommand\E{\mathbb{E}}
    \newcommand\R{\mathbb{R}}
    
    
    \begin{document}
    
    \begin{proposition}
        $f(x) = a_nx^n + ... + a_0 \in \Z[x]$
        p = prime.

        Assume that $p \vert a_n, p \vert a_0,....a_{n-1}$
        and $p^2 \not\vert a_0$

        Then $f(x)$ is irred over $\Q$
    \end{proposition}
    
    \begin{proof}
        Suppose for a contradicition a_nx^n + .. a_0 = (b_rx^r _ ... + b_0)(c_sx^s+...+x_0), with $r,s < n$.

        By Gauss's lemma it is enough to assume that $b_i, c_i \in \Z$.

        $a_0 = b_0c_0$. 

        $\therefore p \vert b_0$ (say) and $p \vert c_0$.

        $a_1 = b_1c_1 + b_1c_0$.
        pdiv .. pdiv    pnot div c_0 

        $\therefore p \vert b_1$

        $\therefore a_{n-1} = b_{n-1}c_o + .... + b_oc_{n-1}$

        $\therefore p \vert b_{n-1}$

        But $r < s$

        $\therefore p \vert b_r$

        $\therefore p \vert b_rc_s = a_n$

        Contradiction!. $\therefore f(x) $ irred. 
    \end{proof}

    \newcommand{\F}{\mathbb{F}}

    Aside: $x^4+1$ is red over $\F_p$ for all $p$, but irred over $\Q$.

    \begin{proposition}
        p prime, then $x^{p-1} + x^{p-2}+ ... + x + 1$ is irreducible over $\Q$. 
        More generally $\frac{x^{p^n}}{x^{p^{n-1}}} = x^{p^{n-1}(p-1)} + x^{p^{n-1}(p-2)} + ... + 1$ is irred over $\Q$. 
    \end{proposition}

    First, two tricks. 

    $\forall c \in F$

    $f(x)$ is irreducible $\iff f(x+c)$ is irreducible. 

    Eisenstein's property

    $f(x) = \lambda x^n$ mod p
    $f(0) \not\cong 0$ mod $p^2$

    Translated version: $g(x) = f(x-c)$. $c \in Z$ 
    
    $g(x) == \lambda(x-c)^n$ mod p. 

    $g(c) \not== 0 mod p^2$ 

    Proof technique: $f(x)$ is Eisenstein wrt p "at x=1"

    i.e. $f(x+1)$ is Eisenstein w.r.t p.

    Important facts:
    \begin{enumerate}
        \item p prime $\implies p \vert {p \choose k}$, for $k = 1,...,p-1$ 
        \item $(x + y)^p = x^p + px^{p-1}y + \dots + p xy^{p-1} + y^p = x^p + y^p$ mod p, as middle terms have $p$ as a factor.  
        \item If $R$ is a Ring of char p, then $F : R \rightarrow R, F(x) = x^p$ is a ring homomorphism (Frobenius homo)
        \item $(x-y)^p == x^p - y^p$ mod $p$
    \end{enumerate}

    \begin{proof}
        $g(x) ?= f(x + 1)$ is Eisenstein wrt p. 

        $g(x) ?= x^{p^{n-1}{p-1}}$ mod p. 

        $g(0) ?\not= 0$ mod p


        i.e. we need

        $f(x+1) ?= xp^{n-1}(p-1)$ mod p. 

        $f(1) = 0$, mod p^2

        Recall $f(x) = \frac{x^{p^n} - 1}{x^{p^{n-1}} - 1} = 1 + x^{p^{n-1}} + ... + x^{p^{n-1}(p-1)}$

        $\therefore f(1) = p \neq 0 $ mod $p^2$

        $x^{p^n} - 1 = (x^{p^{n-1}} - 1)f(x)$. 

        Working mod p,

        $(x - 1)^{p^n} == (x-1)^{p^{n-1}}f(x)$.

        I.e. $\F_p[x]$, $(x-1)^{p^n} = (x-1)^{p^{n-1}}\bar f(x)$

        We get $\therefore f(x) = x-1$
        i.e $f(x) = x-1 mod p$
        $\therefore f(x+1)  $ is Eisenstein w.r.t p. 

        $\therefore f(x)$ is irred / $\Q$. 

        Ex: Prove $f(x+1)$ is Eisenstein directly. 

        Cor. $f(x)$ is the minimal polynomial over $Q$ of $\zeta_p^n = e^{\frac{2\pi i}{p^n}}$

        Cor. $[\Q(\zeta_p^n):\Q] = p^{n-1}(p-1)$

        $[\Q(\zeta_p) : \Q] = p - 1$

        Later : $[\Q(\zeta_m) : \Q] = \phi(m)$
    \end{proof}
    \end{document}