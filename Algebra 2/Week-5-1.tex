\documentclass[twoside]{article}
    \setlength{\oddsidemargin}{0.25 in}
    \setlength{\evensidemargin}{-0.25 in}
    \setlength{\topmargin}{-0.6 in}
    \setlength{\textwidth}{6.5 in}
    \setlength{\textheight}{8.5 in}
    \setlength{\headsep}{0.75 in}
    \setlength{\parindent}{0 in}
    \setlength{\parskip}{0.1 in}
    
    %
    % ADD PACKAGES here:
    %
    
    \usepackage{amsmath,amsfonts,amssymb,graphicx,mathtools,flexisym}
    
    %
    % The following commands set up the lecnum (lecture number)
    % counter and make various numbering schemes work relative
    % to the lecture number.
    %
    \newcounter{lecnum}
    \renewcommand{\thepage}{\thelecnum-\arabic{page}}
    \renewcommand{\thesection}{\thelecnum.\arabic{section}}
    \renewcommand{\theequation}{\thelecnum.\arabic{equation}}
    \renewcommand{\thefigure}{\thelecnum.\arabic{figure}}
    \renewcommand{\thetable}{\thelecnum.\arabic{table}}
    \newcommand{\N}{\mathbb{N}}
    \newcommand{\PS}{\mathcal P}
    \newcommand{\Z}{\mathbb{Z}}
    \newcommand{\Q}{\mathbb{Q}}
    \newcommand{\C}{\mathbb{C}}
    
    
    
    %
    % The following macro is used to generate the header.
    %
    
    %
    % Convention for citations is authors' initials followed by the year.
    % For example, to cite a paper by Leighton and Maggs you would type
    % \cite{LM89}, and to cite a paper by Strassen you would type \cite{S69}.
    % (To avoid bibliography problems, for now we redefine the \cite command.)
    % Also commands that create a suitable format for the reference list.
    \renewcommand{\cite}[1]{[#1]}
    \def\beginrefs{\begin{list}%
            {[\arabic{equation}]}{\usecounter{equation}
             \setlength{\leftmargin}{2.0truecm}\setlength{\labelsep}{0.4truecm}%
             \setlength{\labelwidth}{1.6truecm}}}
    \def\endrefs{\end{list}}
    \def\bibentry#1{\item[\hbox{[#1]}]}
    
    %Use this command for a figure; it puts a figure in wherever you want it.
    %usage: \fig{NUMBER}{SPACE-IN-INCHES}{CAPTION}
    \newcommand{\fig}[3]{
                \vspace{#2}
                \begin{center}
                Figure \thelecnum.#1:~#3
                \end{center}
        }
    % Use these for theorems, lemmas, proofs, etc.
    \newtheorem{theorem}{Theorem}[lecnum]
    \newtheorem{lemma}[theorem]{Lemma} 
    \newtheorem{proposition}[theorem]{Proposition}
    \newtheorem{claim}[theorem]{Claim}
    \newtheorem{corollary}[theorem]{Corollary}
    \newtheorem{definition}[theorem]{Definition}
    \newtheorem{example}[theorem]{Example}
    
    \newenvironment{proof}{{\bf Proof:}}{\hfill\rule{2mm}{2mm}}
    
    % **** IF YOU WANT TO DEFINE ADDITIONAL MACROS FOR YOURSELF, PUT THEM HERE:
    
    \newcommand\E{\mathbb{E}}
    \newcommand\R{\mathbb{R}}
    

    \usepackage{tikz-cd}
    
    \begin{document}

    \begin{theorem}
        If the regular $n-gon$ is constructible, then $n$ is of the form $2^kp_1...p_r$, where the $p_1...p_r$ are distinct primes more than a power of 2.
    \end{theorem}

    \begin{definition}
        Fermat Primes = $p = 1 + 2^j$

        $\frac{x^{\text{odd}\times m} + 1}{x^m + 1} \in \Z[x]$
        
        $\implies p = 1 + 2^{2^s}$
        $p = 3,5,17,2^8, 65537$
        $2^1, 2^2, 2^4, 2^8, 2^{16}$
    \end{definition}

    E.g Regular 7-gon is not constructible.

    Remark: The converse is also true. 

    constructible $n$s = $3,4,5,6,8,10,12,15,16,17$

    \begin{proof}
        n-gon is constructible and  $m \vert n \implies m-gon$ constructible. Therefore it's enough to show 

        1) if $p$ is an odd prime, and the $p-gon$ is constructible $\implies$ p is Fermat.

        2) if p is prime and $p^2$-gon is constructible $\implies$ p = 2

        1) $K = \Q (cos \frac{2\pi}{p}, sin \frac{2\pi}{p})$

        $p-gon$ constructible $\implies [K : \Q] = \text{power of 2}$

        \begin{tikzcd}
        \R \arrow[d, dash] & \arrow[dl, dash, "2"]  K (i) \arrow[dr, dash, "?"] \\
        \arrow[dr, dash, "2^?"] K & & \Q(\zeta_p) \arrow[dl, dash, "p-1"] \\
        &\Q 
        \end{tikzcd}

        $\therefore p-1 \vert 2^{?}$

        $\therefore p-1 $ is a power of 2. 

        $\therefore p$ is a fermat prime.

        2) $\zeta_p^2 $ is a root of $\frac{x^{p^2} - 1}{x^p - 1} = 1 + x^p + .. + x^{p(p-1)}$

        \begin{tikzcd}
            & \arrow[dl, dash, "2"]  K (i) \arrow[dr, dash, "?"] \\
            \arrow[dr, dash, "2^?"] K = \Q(cos(\frac{2\pi}{p^2}), sin(\frac{-2\pi}{p^2})) & & \Q(\zeta_{p^2}) \arrow[dl, dash, "p-1"] \\
            &\Q 
            \end{tikzcd}
    
        $\therefore p(p-1) \vert 2^2$

        $\therefore p = 2$
    \end{proof}

    \section{Splitting Fields}

    \begin{definition}
        F = field,

        $f(x) \in F[x]$, non-zero. 

        An extension $E/F$ is a splitting field of $f(x)$ if 
        \begin{enumerate}
            \item f(x) splits over $E$, i.e. $f(x) = c \Pi_{i=1}^n(x - \alpha_i)$, for some $c, \alpha_i \in E$ 
            \item E is minimal w.r.t this property. i.e. $E = F(\alpha_1,...,\alpha_n)$
        \end{enumerate}
    \end{definition}

    E.g. 
    \begin{itemize}
        \item If $F \subset \C$, then  

        $E = F(\alpha_1, ..., \alpha_n)$, where $\alpha_1, ..., \alpha_n$ are the roots of $f(x) \in \C$, is the unique splitting field for $f(x)$ contained in $\C$

        Idea: Adjoin all the roots of $f(x)$ to $F$.
        \item $f(x) = ax^2 + bx + c$ then 
        $F(\sqrt{b^2 - 4ac})$ is a splitting field. 
    \end{itemize}


    \end{document}

