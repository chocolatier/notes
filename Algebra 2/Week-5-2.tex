\documentclass[twoside]{article}
    \setlength{\oddsidemargin}{0.25 in}
    \setlength{\evensidemargin}{-0.25 in}
    \setlength{\topmargin}{-0.6 in}
    \setlength{\textwidth}{6.5 in}
    \setlength{\textheight}{8.5 in}
    \setlength{\headsep}{0.75 in}
    \setlength{\parindent}{0 in}
    \setlength{\parskip}{0.1 in}
    
    %
    % ADD PACKAGES here:
    %
    
    \usepackage{amsmath,amsfonts,amssymb,graphicx,mathtools,flexisym}
    
    %
    % The following commands set up the lecnum (lecture number)
    % counter and make various numbering schemes work relative
    % to the lecture number.
    %
    \newcounter{lecnum}
    \renewcommand{\thepage}{\thelecnum-\arabic{page}}
    \renewcommand{\thesection}{\thelecnum.\arabic{section}}
    \renewcommand{\theequation}{\thelecnum.\arabic{equation}}
    \renewcommand{\thefigure}{\thelecnum.\arabic{figure}}
    \renewcommand{\thetable}{\thelecnum.\arabic{table}}
    \newcommand{\N}{\mathbb{N}}
    \newcommand{\PS}{\mathcal P}
    \newcommand{\Z}{\mathbb{Z}}
    \newcommand{\Q}{\mathbb{Q}}
    \newcommand{\C}{\mathbb{C}}
    
    
    
    %
    % The following macro is used to generate the header.
    %
    
    %
    % Convention for citations is authors' initials followed by the year.
    % For example, to cite a paper by Leighton and Maggs you would type
    % \cite{LM89}, and to cite a paper by Strassen you would type \cite{S69}.
    % (To avoid bibliography problems, for now we redefine the \cite command.)
    % Also commands that create a suitable format for the reference list.
    \renewcommand{\cite}[1]{[#1]}
    \def\beginrefs{\begin{list}%
            {[\arabic{equation}]}{\usecounter{equation}
             \setlength{\leftmargin}{2.0truecm}\setlength{\labelsep}{0.4truecm}%
             \setlength{\labelwidth}{1.6truecm}}}
    \def\endrefs{\end{list}}
    \def\bibentry#1{\item[\hbox{[#1]}]}
    
    %Use this command for a figure; it puts a figure in wherever you want it.
    %usage: \fig{NUMBER}{SPACE-IN-INCHES}{CAPTION}
    \newcommand{\fig}[3]{
                \vspace{#2}
                \begin{center}
                Figure \thelecnum.#1:~#3
                \end{center}
        }
    % Use these for theorems, lemmas, proofs, etc.
    \newtheorem{theorem}{Theorem}[lecnum]
    \newtheorem{lemma}[theorem]{Lemma} 
    \newtheorem{proposition}[theorem]{Proposition}
    \newtheorem{claim}[theorem]{Claim}
    \newtheorem{corollary}[theorem]{Corollary}
    \newtheorem{definition}[theorem]{Definition}
    \newtheorem{example}[theorem]{Example}
    
    \newenvironment{proof}{{\bf Proof:}}{\hfill\rule{2mm}{2mm}}
    
    % **** IF YOU WANT TO DEFINE ADDITIONAL MACROS FOR YOURSELF, PUT THEM HERE:
    
    \newcommand\E{\mathbb{E}}
    \newcommand\R{\mathbb{R}}
    

    \usepackage{tikz-cd}
    
    \begin{document}

    \subsection{Splitting Fields}

    \begin{definition}
        F field, $f(x) \in F[x]$ non-zero, a splitting field of $f$ is a field extension $E/F$ such that $f(x) = \alpha \Pi_i(x - \alpha_i)$, with $\alpha, \alpha_1, \alpha_2, ..., \alpha_n$ and $E = F(\alpha_1, ..., \alpha_n)$
    \end{definition}

    F field, $A,B$ rings [e.g. $A = F[x]$, $F \rightarrow F[X]$]
    
    \begin{tikzcd}
        A \arrow[d, "i_A"] \arrow[r, "\phi"] & B \\ 
        F \arrow[ur, "i_B"]
    \end{tikzcd}

    \newcommand{\inj}{\rightarrow}
    
    $Hom_F(A,B) = \{\phi : A \rightarrow B\ \vert i_b = \phi i_A\}$

    \begin{proposition}
        $F \inj F[x], F \rightarrow B$ any ring morphism. 

        $Hom_F(F[x], B) \rightarrow[\cong] B$

        $\phi \mapsto \phi(x)$

        \begin{proof}
            Given $b \in B$, define $\phi_b: F[x] \rightarrow B$ by $\phi(\sum_{n=0}^ma_nx^n) = \sum_{n=0}^ma_nb^n$

            Check $\phi_b$ is a ring morphism
        \end{proof}
    \end{proposition}

    Cor: Fix $f(x) \in F[X]$, then there is a bijection $Hom_F (\frac{F[x]}{f(x)}, B)$
    
    TODO: Turn scratchwork into proof.

    Scratchwork below [Also refer to video]

    \begin{tikzcd}
        A \arrow[d, "P"] \arrow[r, "\bar \phi_P"] & B \\ 
        A/I \arrow[ur, "\bar \phi"]
    \end{tikzcd}

    $\phi(a + I) = \phi(a)$ 

    $a + I = a'+I \implies a - a' \in I$

    So $\phi(a) = \phi(a')$

    $A = F[x]$

    $\downarrow$ 
    
    $\frac{F[X]}{f} \rightarrow B$

    % \begin{tikzcd}
    %     \{\phi A \rightarrow B \vert \phi(I) = 0\} \arrow[d, \cong] \\
    %     \{\bar \phi : \frac{A}{I} \rightarrow B\}
    % \end{tikzcd}

    \begin{corollary}
        TODO: Write up from notes
    \end{corollary}

    \begin{proof}
        $f(\alpha) = \frac{F[x]}{(f(x))}$, f min. poly of $\alpha$
    \end{proof}

    Cor: ANy two splitting fields $\frac{E_1}{F}$, and $E_2/F$ of a poly $f(x) \in F[x]$ are $F$-isomorphic. 

    \begin{proof}
        ETS there is an $F-$morphism $\phi:E_1 \rightarrow E_2$. Since then $[E_1:F] \leq [E_2:F]$
        By symmetry there would be a map from $E_2$ to $E_1$, so $[E_1:F] \geq [E_2:F]$

        So $\phi$ will be an isomorphism. 

        Let $\alpha_1, ..., \alpha_n$ be the roots in $E_1$ of f, so $F(\alpha_1, ..., \alpha_n)$.

        Assume by induction we ahve $\phi_i: F(\alpha_1, .., \alpha_i) \rightarrow E_2$

        $F(\alpha_1,..., \alpha_{i+1}) \supseteq F(\alpha_1, ...,\alpha_{i}) \rightarrow E_2$

        Let $g(x)$ be the min poly of $\alpha_{i+1}$ over $F(\alpha_1, ..., \alpha_i)$, then $g \vert F$. So there exists a root of $g \in E_2$, since $F$ splits there. 

        User cor. to define $\phi_{i+1}$
    \end{proof}
    \subsection{Computing the degree of a splitting field}

    \begin{enumerate}
        \item $f(x) = x^3 -2$ over $\Q$.

        $E = \Q()$
    \end{enumerate}

\end{document}


