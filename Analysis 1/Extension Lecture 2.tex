\documentclass[twoside]{article}
\setlength{\oddsidemargin}{0.25 in}
\setlength{\evensidemargin}{-0.25 in}
\setlength{\topmargin}{-0.6 in}
\setlength{\textwidth}{6.5 in}
\setlength{\textheight}{8.5 in}
\setlength{\headsep}{0.75 in}
\setlength{\parindent}{0 in}
\setlength{\parskip}{0.1 in}

%
% ADD PACKAGES here:
%

\usepackage{amsmath,amsfonts,amssymb,graphicx,mathtools,flexisym}

%
% The following commands set up the lecnum (lecture number)
% counter and make various numbering schemes work relative
% to the lecture number.
%
\newcounter{lecnum}
\renewcommand{\thepage}{\thelecnum-\arabic{page}}
\renewcommand{\thesection}{\thelecnum.\arabic{section}}
\renewcommand{\theequation}{\thelecnum.\arabic{equation}}
\renewcommand{\thefigure}{\thelecnum.\arabic{figure}}
\renewcommand{\thetable}{\thelecnum.\arabic{table}}
\newcommand{\N}{\mathbb{N}}
\newcommand{\PS}{\mathcal P}

%
% The following macro is used to generate the header.
%

%
% Convention for citations is authors' initials followed by the year.
% For example, to cite a paper by Leighton and Maggs you would type
% \cite{LM89}, and to cite a paper by Strassen you would type \cite{S69}.
% (To avoid bibliography problems, for now we redefine the \cite command.)
% Also commands that create a suitable format for the reference list.
\renewcommand{\cite}[1]{[#1]}
\def\beginrefs{\begin{list}%
        {[\arabic{equation}]}{\usecounter{equation}
         \setlength{\leftmargin}{2.0truecm}\setlength{\labelsep}{0.4truecm}%
         \setlength{\labelwidth}{1.6truecm}}}
\def\endrefs{\end{list}}
\def\bibentry#1{\item[\hbox{[#1]}]}

%Use this command for a figure; it puts a figure in wherever you want it.
%usage: \fig{NUMBER}{SPACE-IN-INCHES}{CAPTION}
\newcommand{\fig}[3]{
			\vspace{#2}
			\begin{center}
			Figure \thelecnum.#1:~#3
			\end{center}
	}
% Use these for theorems, lemmas, proofs, etc.
\newtheorem{theorem}{Theorem}[lecnum]
\newtheorem{lemma}[theorem]{Lemma}
\newtheorem{proposition}[theorem]{Proposition}
\newtheorem{claim}[theorem]{Claim}
\newtheorem{corollary}[theorem]{Corollary}
\newtheorem{definition}[theorem]{Definition}
\newenvironment{proof}{{\bf Proof:}}{\hfill\rule{2mm}{2mm}}

% **** IF YOU WANT TO DEFINE ADDITIONAL MACROS FOR YOURSELF, PUT THEM HERE:

\newcommand\E{\mathbb{E}}

\begin{document}
%FILL IN THE RIGHT INFO.
%\lecture{**LECTURE-NUMBER**}{**DATE**}{**LECTURER**}{**SCRIBE**}
% \lecture{2}{March 03}{Analysis Extension}{Axiomatic Set Theory}
%\footnotetext{These notes are partially based on those of Nigel Mansell.}

% **** YOUR NOTES GO HERE:
\section{Axioms of ZFC (Continued)}

\begin{enumerate}
    \setcounter{enumi}{6}
    \item Axiom of Specification:

    The collection of all elements of a known set, that satisfiy a certain predicate, is a set.

    E.g $\{x \in \N \vert x \text{ is prime}\}$
    
    \item Axiom of power set

    If S is a set, then the collection of all subsets of S is a set, called the power set. $\mathcal P (S)$

    $(a,b) = \{\{a\}, \{a, b\}\}$

    $A \times B := \text{set of ordered pairs} (a,b) = \{\{a\}, \{a, b\}\}$

    Let $A,B$ be sets. A function $f : A \rightarrow B$ is an element of the $\mathcal{P} (A \times B)$,

    \begin{itemize}
        \item For every $a \in A$, there is some $b \in B$ such that $(a,b) \in F$
        
       \item if $b_1$ and $b_2 \in B$ such that $(a,b_1) \in \in F$ and $(a, b_2) \in f$ then $b_1 = b_2$
    \end{itemize}

    \item Let $X$ be a set. Define $\succ(x) = x^{+} := x \cup \{x\}$.
    There exists a set S such that
    \begin{enumerate}
        \item $\emptyset \in S$, 
        \item $x \in S$ then $x^{+} \in S$.
        
        E.g. $succ(\emptyset) = \emptyset \cup {\emptyset} = {\emptyset}$
        $\{\emptyset\}^{+} = \{\emptyset\} \cup \{\{\emptyset\}\} = \{\emptyset, \{\emptyset\}\}$
    \end{enumerate}

    \item Axiom of choice. 

    A choice function f defined on a set X of non-empty sets is a function with the property that if $a \in X$ then $f(a) \in a$.

    Choice functions always exist for any $X$.
\end{enumerate}


\section{Construction  of $\N$}

$\N$ includes 0.

Define $0 := \emptyset$

$1 := \emptyset^+$


$2 := 1^+ = \{\emptyset, \{\emptyset\}\}$

$3 := 2^+ = \{\emptyset, \{\emptyset\}\} \cup \{\{\emptyset, \{ \emptyset\}\}\} = \{\emptyset, \{\emptyset\},\{\emptyset, \{ \emptyset\}\}\}$

Q: What is $\N$.

A9 tells us that there is a set $S$ such that $\phi \in s,$ if $x \in S, x^+ \in S$

Consider $I_S := \{T \in \PS (S) \vert \emptyset \in T \text{ and } \forall x \in T, x^+ \in T\}$

Say that a set $A$ is inductive. If $\emptyset \in A \text{ and if} x \in A, x^+ \in A$.

$I_s =$ set of all inductive subsets of S.

$I_s \neq \emptyset$ because $S \in I_S$

Define $\N = \cap_{x \in I_S} x = \{ x \in S \vert \forall T \in I_s, x \in T\}$

\begin{theorem}{Principle of Mathematical Induction}

    Let $p$ be a predicate defined on $\N$. Assume that $p(0)$ holds, and for every $n \in \N, p(n)$ implies $p (n^+)$.
    Then $p(n)$ holds for every $n \in N$.
    
\end{theorem}

\begin{proof}
    Fix p, with the above properties. 

    $S = \{x \in \N \vert p(x) \text{ holds}\}$.
    We want to show that $S = N$, i.e. the elements of $S$ are exactly the elements of $\N$. 

    Observation
    S is inductive
    \begin{enumerate}
        \item $\emptyset \in S$, because $p(0) = p(\emptyset)$ holds. 
        \item If $x \in S$ it means $p(x)$ holds, therefore $p(x^+)$ also holds. 

        So, $x^+ \in S$  
    \end{enumerate}

    1 and 2 tell us that $S$ is inductive. 

    $S \subseteq \N$ by construction. 

    $N \subseteq S$ by construction. 

    $\implies S = \N$
\end{proof}

\begin{theorem}
    If $m, n \in \N$, such that $m^+ = n^+$, then $m = n$. 
\end{theorem}

\begin{proof}
    \begin{lemma}
        Let $x, n \in \N$.
        If $x \in n$ then $x \subseteq n$ 
    \end{lemma}
\end{proof}
\end{document}