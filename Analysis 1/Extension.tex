\documentclass[twoside]{article}
    \setlength{\oddsidemargin}{0.25 in}
    \setlength{\evensidemargin}{-0.25 in}
    \setlength{\topmargin}{-0.6 in}
    \setlength{\textwidth}{6.5 in}
    \setlength{\textheight}{8.5 in}
    \setlength{\headsep}{0.75 in}
    \setlength{\parindent}{0 in}
    \setlength{\parskip}{0.1 in}
    
    %
    % ADD PACKAGES here:
    %
    
    \usepackage{amsmath,amsfonts,amssymb,graphicx,mathtools,flexisym}
    
    %
    % The following commands set up the lecnum (lecture number)
    % counter and make various numbering schemes work relative
    % to the lecture number.
    %
    \newcounter{lecnum}
    \renewcommand{\thepage}{\thelecnum-\arabic{page}}
    \renewcommand{\thesection}{\thelecnum.\arabic{section}}
    \renewcommand{\theequation}{\thelecnum.\arabic{equation}}
    \renewcommand{\thefigure}{\thelecnum.\arabic{figure}}
    \renewcommand{\thetable}{\thelecnum.\arabic{table}}
    \newcommand{\N}{\mathbb{N}}
    \newcommand{\PS}{\mathcal P}
        
    %Use this command for a figure; it puts a figure in wherever you want it.
    %usage: \fig{NUMBER}{SPACE-IN-INCHES}{CAPTION}
    \newcommand{\fig}[3]{
                \vspace{#2}
                \begin{center}
                Figure \thelecnum.#1:~#3
                \end{center}
        }
    % Use these for theorems, lemmas, proofs, etc.
    \newtheorem{theorem}{Theorem}[lecnum]
    \newtheorem{lemma}[theorem]{Lemma}
    \newtheorem{proposition}[theorem]{Proposition}
    \newtheorem{claim}[theorem]{Claim}
    \newtheorem{corollary}[theorem]{Corollary}
    \newtheorem{definition}[theorem]{Definition}
    \newenvironment{proof}{{\bf Proof:}}{\hfill\rule{2mm}{2mm}}
    
    % **** IF YOU WANT TO DEFINE ADDITIONAL MACROS FOR YOURSELF, PUT THEM HERE:
    
    \newcommand\E{\mathbb{E}}
    
    \begin{document}
    \section {Set Theory}

    \begin{definition}{(bad)}

        A set is an object $S$ such that for every $x$, we have exactly one of two things.

        \begin{enumerate}
            \item $x \in S$ ($x$ belongs to $S$) or 
            \item $x \not\in S$ ($x$ does not belong to $S$) 
        \end{enumerate}
    \end{definition}

    Sets are specified as follows
    \begin{enumerate}
        \item $\{1,2,3,4\}$
        \item $\{1,2,3,4,...\}$
        \item $\{x \in \N \vert x \text{ is prime}\}$
    \end{enumerate}

    \subsection{Russel's Paradox}
        Let $S$ be the following:

        $S = \{T \vert T \not\in T\}$

        Q: Is $S \in S?$

        If yes, then by definition $S \not\in S$
        If no, then by definition $S \in S$

    \subsection{Foundations of Set Theory/Zermelo-Frankel Set Theory}
    \subsubsection{Axioms of ZFC}

    \begin{enumerate}
        \item Axiom of Extension: 
        Two sets are equal if and only if they have the same elements. 

        \item Axiom of Existence:
        The Empty Set Exists

        \item Axiom of Pairing:
        If $X$ and $Y$ are sets, then there is a set $\{X, Y\}$

        Example : $X = \{1,2\}$         $Y = \{3,4\}$
        ${X,Y} = \{\{1,2\}, \{3,4\}\}$

        $X = \emptyset$, $Y = \emptyset$

        $\{X,Y\} = \{\emptyset\}$

        \item Axiom of Union:        If $S$ is a (set of sets), then the union over elements of $S$ is a set. 

        Example: $S = \{\{1,2\}, \{2,3\}\}$, then $\{1,2,3\}$ is a set. 

        \item Axiom of Intersection: Redundant, as above. 

        \item Axiom of foundation.

        Every $x \neq \emptyset$ contains a member $y$ such that $x \neq y$. 

        Consequence of failure:

        Let $X \neq \emptyset$

        $\forall y \in X, y \cap X \neq \emptyset$

        Choose any $y_1 \in X$, choose $y_2 \in X \cap Y$. $y_2 \in X$, so $y_2 \cap X \neq \emptyset$. Choose $y_3 \in x \cap y_2$...

        \item Axiom Schema of Replacement

        If $\phi$ is a formula of 1st order logic, $A$ is any set, then $\phi$ applied to $A$ is contained in a set. 

    \end{enumerate}
    
    \end{document}