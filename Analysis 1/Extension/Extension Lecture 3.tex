\documentclass[twoside]{article}
\setlength{\oddsidemargin}{0.25 in}
\setlength{\evensidemargin}{-0.25 in}
\setlength{\topmargin}{-0.6 in}
\setlength{\textwidth}{6.5 in}
\setlength{\textheight}{8.5 in}
\setlength{\headsep}{0.75 in}
\setlength{\parindent}{0 in}
\setlength{\parskip}{0.1 in}

%
% ADD PACKAGES here:
%

\usepackage{amsmath,amsfonts,amssymb,graphicx,mathtools,flexisym}

%
% The following commands set up the lecnum (lecture number)
% counter and make various numbering schemes work relative
% to the lecture number.
%
\newcounter{lecnum}
\renewcommand{\thepage}{\thelecnum-\arabic{page}}
\renewcommand{\thesection}{\thelecnum.\arabic{section}}
\renewcommand{\theequation}{\thelecnum.\arabic{equation}}
\renewcommand{\thefigure}{\thelecnum.\arabic{figure}}
\renewcommand{\thetable}{\thelecnum.\arabic{table}}
\newcommand{\N}{\mathbb{N}}
\newcommand{\PS}{\mathcal P}
\newcommand{\Z}{\mathbb{Z}}


%
% The following macro is used to generate the header.
%

%
% Convention for citations is authors' initials followed by the year.
% For example, to cite a paper by Leighton and Maggs you would type
% \cite{LM89}, and to cite a paper by Strassen you would type \cite{S69}.
% (To avoid bibliography problems, for now we redefine the \cite command.)
% Also commands that create a suitable format for the reference list.
\renewcommand{\cite}[1]{[#1]}
\def\beginrefs{\begin{list}%
        {[\arabic{equation}]}{\usecounter{equation}
         \setlength{\leftmargin}{2.0truecm}\setlength{\labelsep}{0.4truecm}%
         \setlength{\labelwidth}{1.6truecm}}}
\def\endrefs{\end{list}}
\def\bibentry#1{\item[\hbox{[#1]}]}

%Use this command for a figure; it puts a figure in wherever you want it.
%usage: \fig{NUMBER}{SPACE-IN-INCHES}{CAPTION}
\newcommand{\fig}[3]{
			\vspace{#2}
			\begin{center}
			Figure \thelecnum.#1:~#3
			\end{center}
	}
% Use these for theorems, lemmas, proofs, etc.
\newtheorem{theorem}{Theorem}[lecnum]
\newtheorem{lemma}[theorem]{Lemma}
\newtheorem{proposition}[theorem]{Proposition}
\newtheorem{claim}[theorem]{Claim}
\newtheorem{corollary}[theorem]{Corollary}
\newtheorem{definition}[theorem]{Definition}
\newtheorem{example}[theorem]{Example}

\newenvironment{proof}{{\bf Proof:}}{\hfill\rule{2mm}{2mm}}

% **** IF YOU WANT TO DEFINE ADDITIONAL MACROS FOR YOURSELF, PUT THEM HERE:

\newcommand\E{\mathbb{E}}
\newcommand\R{\mathbb{R}}


\begin{document}

\subsection{recap}

Defined $\N$

$0 = \emptyset$

$1 = \{\emptyset\}$

$2 = \{\emptyset, \{\emptyset\}\}$

\subsection{Today}

Arithmetic on $\N$

How can we make sense of "1+2 = 3?"


Strategy: Define the function "addition of m" for every $m \in N$

\begin{theorem}
    For every $m \in \N$, there exists a function $S_m \subset \N \times \N$ such that 
    \begin{itemize}
        \item     $S_m(0) = m$
        \item $\forall n \in \N, S_m(n^+) = [S_m(n)]^+$
    \end{itemize}
\end{theorem}

\begin{proof}
    Let's construct a relation $S_m' = \N \times \N$ with the "right" properties, then we'll show $S_m'$ is a function.

    $\mathcal{S}_m = \{R \subset \N \times \N \vert (0,m) \in R, \text{ and if } (n,x) \in R \text{ then } (n^+,x^+) \in R\}$

    $\mathcal{S}_m \ni \N \times \N$, so $\mathcal{S} \neq \emptyset$

    Define $s_m' = \cap_{R\in\mathcal{S_m}} R$

    $Notice:$
    \begin{enumerate}
        \item $S+m' \subseteq \N \times \N$

        \item $(0,m) \in s_m'$

        \item If $(n,x) \in s_m', then (n^+,x^+) \in s_m'$

    \end{enumerate}

    Strategy: Show that $S_m'$ is a function.

    Show

    \begin{enumerate}
        \item $Domain (s_m') = \N$
        \item If $(n,x)$ and $(n,y)$ are both in $s_m'$ then then $x = y$
    \end{enumerate}

    \begin{proof} of 1

        Let $p(n)$ be true if there is some $x \in \N$ such that $(n,x) \in s'_m$
        
        $p(0)$ is true because $(0,m) \in s_m'$

        Suppose $p(n)$ is true, so there is some $x \in \N$ such that $(n,x) \in s_'m \implies (n^+, x^+)$, so $p_n$ is true. 

        So domain $s_m'= \N$ 
        
    \end{proof}

    \begin{proof} of 2
        Let $p(n)$ be true if for $\forall r \in n$, there is a unique $x \in \N$ such that $(r,x)\in s_m'$.

        $p(0)$ is vacuously true. 

        Suppose $p(n)$ is true. 

        We want to show $p(n^+)$, i.e. $\forall r \in \N$, $r \in n+, \exists! x \in \N$ such that $(r,x) \in s_m'$

        If $r \in n^+$, then either $r \in n$ or $r = n$. Consider any $r \in n^+$.

        If $R \in n$, then we know $\exists! x \in \N$, s.t. $(r,x) \in s_m'$

        Now lets prove the statement for $r = n$. 

        Suppose that $\exists a,b \in \N$ such that $(n,a)$ and $(n,b)$ are both in $s_m'$.

        Claim: If $a \neq b$, not both of these can have "predecessors". 

        Suppose $a \neq b$

        Suppose that there are $p_1,x_1,p_2,x_2 \in \N$ such that $(p_1,x_1) \in s_m', (p_2,x_2) \in s_m' and (p_1,x_1) = (n,a), (p_2,x_2) = (n,b)$

        $\implies p_1^{+} = p_2{+}, \implies p_1 = p_2 = p$

        $(p,x_1) \in s_m', (p,x_2) \in s_m'$ and $p^+ = n$ so $p \in n$. 

        Induction hypothesis $x_1 = x_2 \implies a = b$. Contradiciton. 

        Suppose $a \neq b$. Suppose WLOG that $(n,a)$ doesn't have a predecessor. Define $s_m'' := s_m'(n,a)$.

        $s_m''$ satisifies the properties of being in $\mathcal{S}_m$ - Exercise. 

        Problem, because $s_m' $ was the intersection of all $R \in \mathcal{S}_m$.

        $\implies s'_m \subseteq s_m''$. Contradiciton. 

        Write $s_m' = s_m$ Addition by m function. 
    \end{proof}

    Properties
    
    \begin{enumerate}
        \item $\forall n \in \N, s_1(n) = n^+$
        \item $\forall n\in \N, s_0(n) = n$
        \item $\forall m,n,k \in \N$ we have $S_{s_m(n)}(k) = s_m(s_n(k))$ (Assoc)
        \item $\forall m,n \in \N, $ we have $s_m(n) = s_n(m)$.

    \end{enumerate}

    \begin{itemize}
        \item Let $p(n)$ be true if $s_1(n) = n^+$, $p(0)$. Suppose $p(n)$ is true. $s_1(n^+) = [s_1(n)]^+ = (n^+)^+$
        $p(n^+)$ is true. 

        
    \end{itemize}


\end{proof}

\end{document}