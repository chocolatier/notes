\documentclass[twoside]{article}
\setlength{\oddsidemargin}{0.25 in}
\setlength{\evensidemargin}{-0.25 in}
\setlength{\topmargin}{-0.6 in}
\setlength{\textwidth}{6.5 in}
\setlength{\textheight}{8.5 in}
\setlength{\headsep}{0.75 in}
\setlength{\parindent}{0 in}
\setlength{\parskip}{0.1 in}

%
% ADD PACKAGES here:
%

\usepackage{amsmath,amsfonts,amssymb,graphicx,mathtools,flexisym}

%
% The following commands set up the lecnum (lecture number)
% counter and make various numbering schemes work relative
% to the lecture number.
%
\newcounter{lecnum}
\renewcommand{\thepage}{\thelecnum-\arabic{page}}
\renewcommand{\thesection}{\thelecnum.\arabic{section}}
\renewcommand{\theequation}{\thelecnum.\arabic{equation}}
\renewcommand{\thefigure}{\thelecnum.\arabic{figure}}
\renewcommand{\thetable}{\thelecnum.\arabic{table}}
\newcommand{\N}{\mathbb{N}}
\newcommand{\PS}{\mathcal P}
\newcommand{\Z}{\mathbb{Z}}

\newcommand{\Q}{\mathbb{Q}}


%
% The following macro is used to generate the header.
%

%
% Convention for citations is authors' initials followed by the year.
% For example, to cite a paper by Leighton and Maggs you would type
% \cite{LM89}, and to cite a paper by Strassen you would type \cite{S69}.
% (To avoid bibliography problems, for now we redefine the \cite command.)
% Also commands that create a suitable format for the reference list.
\renewcommand{\cite}[1]{[#1]}
\def\beginrefs{\begin{list}%
        {[\arabic{equation}]}{\usecounter{equation}
         \setlength{\leftmargin}{2.0truecm}\setlength{\labelsep}{0.4truecm}%
         \setlength{\labelwidth}{1.6truecm}}}
\def\endrefs{\end{list}}
\def\bibentry#1{\item[\hbox{[#1]}]}

%Use this command for a figure; it puts a figure in wherever you want it.
%usage: \fig{NUMBER}{SPACE-IN-INCHES}{CAPTION}
\newcommand{\fig}[3]{
			\vspace{#2}
			\begin{center}
			Figure \thelecnum.#1:~#3
			\end{center}
	}
% Use these for theorems, lemmas, proofs, etc.
\newtheorem{theorem}{Theorem}[lecnum]
\newtheorem{lemma}[theorem]{Lemma}
\newtheorem{proposition}[theorem]{Proposition}
\newtheorem{claim}[theorem]{Claim}
\newtheorem{corollary}[theorem]{Corollary}
\newtheorem{definition}[theorem]{Definition}
\newtheorem{example}[theorem]{Example}

\newenvironment{proof}{{\bf Proof:}}{\hfill\rule{2mm}{2mm}}

% **** IF YOU WANT TO DEFINE ADDITIONAL MACROS FOR YOURSELF, PUT THEM HERE:

\newcommand\E{\mathbb{E}}
\newcommand\R{\mathbb{R}}


\begin{document}

\subsection{recap}

Construction of $\R$.
$R = $ set of all cauchy sequences of rational numbers. 

We say that $(a_n) \text{\textasciitilde} (b_n)$ if their difference $(a_n - b_n)$ converges to 0. 
$\R := R/\text{\textasciitilde}  =$ set of equivalence classes.

\begin{lemma}
    Every cauchy sequence is bdd, i.e. there is some M s.t. $|a_n| \leq M$     
\end{lemma}

\begin{lemma}
    Every Cauchy sequene (a_n) that doesn't converge to 0, is bounded away from 0, i.e. $\exists \varepsilon > 0 and N \in \N$ s.t. $\forall n > N, |a_n|>\varepsilon$. 
\end{lemma}

\begin{proposition}[\textasciitilde is an equivalence relation]

    \begin{enumerate}
        \item Reflexivity : $(a_n) ~ (a_n)$
        \item Symmetry :         $(a_n) ~ (b_n)$ then $a_n - b_n \rightarrow 0$

        $b_n - a_n \rightarrow 0$

        so $(b_n) \text{\textasciitilde} (a_n)$
        \item Transitivity: Suppose $(a_n) ~ (b_n)$ and $(b_n) ~ (c_n)$. 

        So $(a_n - b_n) \rightarrow 0$ and $(b_n - c_n) \rightarrow 0$

        $\forall \varepsilon > 0 \exists N_1 \in \N$ s.t. if $n > N_1$, then $|a_n - b_n| < \frac{\varepsilon}{2}$
        $\forall \varepsilon > 0 \exists N_2 \in \N$ s.t. if $n > N_2$, then $|b_n - c_n| < \frac{\varepsilon}{2}$

        $N = \max{N_1,N_2} and n > N$. 

        $|a_n - c_n| \leq |a_n-b_n| + |b_n - c_n| < \varepsilon$. 
    \end{enumerate}

\end{proposition}

Define $\R = $ set of real numbers $:= R/~$. Elements of $\R$ are $[(a_n)]$

\subsection{p-adics}

$p \in \N$ prime. 

We can define a new metric on $\Q$. Let $x \in \Q$. Write $x$ uniquely as $x = p^a\frac{r}{s}$, where $r,s,a \in \Z$. 
and $p \not\vert r$ and $p \not\vert s$. 
\begin{example}
    p = 3, $x = \frac{7}{6}, x = (3^{-1})(\frac72)$. $|x|_3 = 3$

    $|x|_p := p^{-a}$.
\end{example}

If $x,y \in \Q$, $|x-y|_p$ is the p-adic distance. 

% $\Q_p$ is field of p-adic numbers. := $\frac{\{\text{Cauchy sequences w.r.t. |.|_p}\}}{\text{Equiv relation (a_n) ~ (b_n) iff a_n - b_n \rightarrow 0}}$

\subsection{$\R$ is well defined.}

$\Q \hookrightarrow \R$
$q \mapsto [(q,q,q,q,q)]$

If $p \neq q$ then $i_p \neq i_q$
bc $[p-q, p-q, p-q,...] \neq [(0,0,0,0,0)]$

in $\R$ we have a $0 := [(0,0,0,0,0)]$ and $1 := [(1,1,1,1,.)]$

Properties.
\begin{enumerate}
    \item Addition: $[(a_n)] + [(b_n)] = [(a_n +_{\Q} b_n)]$
    a) Well defined. 

    b) $(a_n) + (b_n)$ is a cauchy seq. 

    c) commutativity, associativity, $0+[(a_n)] = [(a_n)]$

    \item Multiplication $[a_n]\cdot_{\R}[(b_n)] := [(a_n \cdot_{\Q} b_n)]$
    a) This is well defined. 

    \begin{proof}
        LEt $[(a_n)] = [(c_n)]$,$[(b_n)] = [(d_n)]$ 
        Want $[(a_nb_n)] = [(c_nd_n)]$. 

        WTS $(a_nb_n - c_nd_n) \rightarrow 0$ 
        We know $a_n - c_n \rightarrow 0$

        We know $b_n - d_n \rightarrow 0$
        $a_nb_n - c_nd_n = a_nb_n - b_nc_n + {b_nc_n - c_nd_n}$

        $= b_n(a_n-c_n) + c_n(b_n-d_n)$. 

        $|a_nb_n - c_nd_n| \leq |b_n||(a_n - c_n)| + |c_n||b_n - d_n|$. 

        Let $M$ be an upper bound for $|b_n|, |c_n|$

        $\forall \varepsilon > 0, \exists N \in \N$ s.t. $\forall n > N$, we have $|a_n - c_n < \frac\varepsilon{2M}|$

        $|b_n - d_n < \frac\varepsilon{2M}|$. 

        $|a_nb_n - c_nd_n| < \varepsilon$. 
    \end{proof}

    \item Division: 
    
    Let $[(a_n)] \in \R$ and $[(a_n) \neq 0]$. There is some $[(b_n)] \in \R$, s.t. $[(a_n)] \cdot [(b_n)] = [(1,1,1,1,1...)] = 1$

    By Lemma 2, If $[(a_n)] \neq 0$ then some $\varepsilon 0$, $\exists N \in \N$, s.t. $\forall n > N$, $|a_n| > \varepsilon$

    $\implies$ eventually $a_n \neq 0$. 

    Define $b_n$ as follows: if $n \leq N$, set $b_n = 1$. If $n > N$, set $b_n = \frac1{a_n}$

    $a_nb_n =$ if $n \leq N$, $a_n$ 

    if $n > N$, 1 

    $\implies [(a_nb_n)] = [(1,1,1,1..)]$

    Prop: $b_n$ is cauchy. 

    $|b_m - b_n| = |\frac1{a_m} - \frac1{a_n}| = |\frac{a_n - a_m}{a_ma_n}|$

    \begin{theorem}
        $\R$ is a field. 
    
        ($\Q_p$ is a field.)     
    \end{theorem}
    
    \item Order Relation 
    
    We say that $[(a_n)] \geq [(b_n)]$ if either $[(a_n)] = [(b_n)]$ or $\exists N \in \N$ s.t. $\forall n > \N$, $a_n - b_n \geq 0$. 

    $Thm: \leq $ is a total oreder on $\R$. 

    \subsection{Completeness of $\R$}

    Since $\R$ is ordered, $Completeness \iff lub property$

\end{enumerate}

\end{document}