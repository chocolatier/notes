\documentclass[twoside]{article}
\setlength{\oddsidemargin}{0.25 in}
\setlength{\evensidemargin}{-0.25 in}
\setlength{\topmargin}{-0.6 in}
\setlength{\textwidth}{6.5 in}
\setlength{\textheight}{8.5 in}
\setlength{\headsep}{0.75 in}
\setlength{\parindent}{0 in}
\setlength{\parskip}{0.1 in}

%
% ADD PACKAGES here:
%

\usepackage{amsmath,amsfonts,amssymb,graphicx,mathtools,flexisym}

%
% The following commands set up the lecnum (lecture number)
% counter and make various numbering schemes work relative
% to the lecture number.
%
\newcounter{lecnum}
\renewcommand{\thepage}{\thelecnum-\arabic{page}}
\renewcommand{\thesection}{\thelecnum.\arabic{section}}
\renewcommand{\theequation}{\thelecnum.\arabic{equation}}
\renewcommand{\thefigure}{\thelecnum.\arabic{figure}}
\renewcommand{\thetable}{\thelecnum.\arabic{table}}
\newcommand{\N}{\mathbb{N}}
\newcommand{\PS}{\mathcal P}
\newcommand{\Z}{\mathbb{Z}}
\newcommand{\TO}{\mathcal O}

\newcommand{\Q}{\mathbb{Q}}


%
% The following macro is used to generate the header.
%

%
% Convention for citations is authors' initials followed by the year.
% For example, to cite a paper by Leighton and Maggs you would type
% \cite{LM89}, and to cite a paper by Strassen you would type \cite{S69}.
% (To avoid bibliography problems, for now we redefine the \cite command.)
% Also commands that create a suitable format for the reference list.
\renewcommand{\cite}[1]{[#1]}
\def\beginrefs{\begin{list}%
        {[\arabic{equation}]}{\usecounter{equation}
         \setlength{\leftmargin}{2.0truecm}\setlength{\labelsep}{0.4truecm}%
         \setlength{\labelwidth}{1.6truecm}}}
\def\endrefs{\end{list}}
\def\bibentry#1{\item[\hbox{[#1]}]}

%Use this command for a figure; it puts a figure in wherever you want it.
%usage: \fig{NUMBER}{SPACE-IN-INCHES}{CAPTION}
\newcommand{\fig}[3]{
			\vspace{#2}
			\begin{center}
			Figure \thelecnum.#1:~#3
			\end{center}
	}
% Use these for theorems, lemmas, proofs, etc.
\newtheorem{theorem}{Theorem}[lecnum]
\newtheorem{lemma}[theorem]{Lemma}
\newtheorem{proposition}[theorem]{Proposition}
\newtheorem{claim}[theorem]{Claim}
\newtheorem{corollary}[theorem]{Corollary}
\newtheorem{definition}[theorem]{Definition}
\newtheorem{example}[theorem]{Example}

\newenvironment{proof}{{\bf Proof:}}{\hfill\rule{2mm}{2mm}}

% **** IF YOU WANT TO DEFINE ADDITIONAL MACROS FOR YOURSELF, PUT THEM HERE:

\newcommand\E{\mathbb{E}}
\newcommand\R{\mathbb{R}}


\begin{document}
\subsection{Note}

Exercise: Subspace topology

$A \subseteq B \subseteq X$

The subspace topology on $A$ coming from $X$ = the subspace topology on $A$ coming from the subspace topology on $B$ coming from $X$. 

\subsection{Product Topology}

Let $(X, \TO_x)$ and $(Y, \TO_y)$ be topological spaces. Then we give the product topology on the set $X \times Y$ as follows:

The basis $B_{x\times y} = \TO_x \times to_y$ and $\TO_x \times to_y =$ all possible unions of sets in $B_{x \times y}$

\begin{example}
    $\R^2 = \R \times \R$ w/ product top. 

    $\triangle \subseteq \R^2; \triangle = \{(x,x) \vert x \in \R\}$

    Prop: $(\R^2 \backslash \triangle)$ is open in $\R^2$, but cannot be written as $U \times V$ for $U,V$ open in $\R$

    E.g. Cylinder $\subset \R^3$ is homeomorphic to Annulus $\subset \R^2$ is homeomorphic to $S^1 \times [0,1]$.     
\end{example}

\begin{definition}
    Connectedness: Defn: A topological space $X$ is called connected if it cannot be written as a union of two non-empty and disjoint open sets. 
\end{definition}
\begin{example}
    $X = [0,1] \cup (3,4) \subset \R$ is disconnected because $[0,1], (3,4)$ both open and disjoint. 
\end{example}

\subsection{Equivalent Properties}

\begin{enumerate}
    \item $X$ is connected iff it cannot be written as the union of two non-empty disjoint closed subsets. 

    If $X$ disconnected, $X = A \cup B$, $A. b \neq \emptyset$, $A \cap B = \emptyset$, $A, B$ $\implies A, B$ are both closed. $A, B$ are clopen.
    
    \item $X$ is connected if and only if the only clopen subsets of $X$ are $\emptyset$ and $X$. 

    If $\emptyset \subsetneq A \subsetneq X$ is clopen, then $(X\backslash A)$ is clopen also. $X = A \cup (X \backslash A)$

    \begin{example}
        $\Q = (\Q \cap (-\infty, \sqrt2)) \cup (\Q \cap (\sqrt2, \infty)) $
    \end{example}
\end{enumerate}

\newcommand{\Rm}{\R_{\text{metric}}}
\newcommand{\ep}{\varepsilon}

Any interval in $\R_{\text{metric}}$ is connected. 


\begin{proof}
    (Proving closed intervals for now)

    Let $X = [a,b]$. If $a = b$, $x = \{a\}$ is connected. 

    Now let $a \neq b$; so $a < b$.

    Suppose $X$ is disconnected. $X = A \cup B$, where $A$ and $B$ are clopen, and $A$ is non-empty. 

    Suppose WLOG that $a \in A$. 

    A is open in $X$, so $A = [a,b] \cap U$, where $U$ open in $\R$. 
    
    $a \subset U$, so $\exists \ep > 0$, such that $(a - \ep, a + \ep) \subseteq U$. 

    $(a - \ep, a+\ep) \cap [a,b] \subset A$ = $[a, a+\ep) \subseteq A$. 

    So $[a, \frac\ep2] \subset A$.

    Let $C := \{c \in [a,b] \vert [a,c] \subseteq A\}$

    $C$ has an upper bound, namely $b$. 

    $C$ has a least-upper bound, $L \leq B$

    $L$ is a limit point of $A$. 

    $L \in X$ because $a < L \leq B$.

    A is clopen in $X$, so $A$ is closed in $X$.

    $\implies L \in A$

    $\implies [a,L] \subset A$

    We'd like to show that $L = b$. 

    $L \in A$

    $\implies$ A being open, $\exists \ep > 0$, s.t. $[L, L + \ep) \subseteq A$.
    
    $\implies$ $[L, L + \frac\ep2] \subseteq A$. 

    Bad, because it implies that $[a, L + \frac\ep2] \subseteq A$, which is a contradiction. $\implies L = B$. 

    Proof $\implies A = X \implies B$ is empty $\implies$ any closed interval is connected in $\Rm$. 
\end{proof}

\begin{definition}
    Connected Component: Let $X$ be a space. 

    $x \in X$. The connected component of $x \in X$ is the union of all connected $Y \subseteq X$ of $X$, s.t. $x \in Y$.

    Remark: The connected component of any $x \in X$ is a connected space. (needs proof)

\end{definition}

\begin{definition}
    Path Connectedness

    A space $X$ is path connected if $\forall a, b \in X$, there is a continuous function $f: [0,1] \in R_m \rightarrow X$ with $f(0) = a$, $f(1) = b$. 
\end{definition}

$[0,1]$ is homeomorphic to $[p,q]$ if $p \neq q$

\begin{proposition}
    If $X$ is path connected, then it is connected. 
\end{proposition}

\begin{proof}
    If not then $X = A \cup B$, disjoint union on non-empty clopen sets. 

    Take $a \in A$, $b \in B$. 

    There exists some $f:[0,1] \rightarrow X$, $f(0)=a$, $f(1) = b$. $[0,1] = f^{-1}(A) \cup f^{-1}(B)$ disjoint union, contradiction $[0,1]$ is connected. 
\end{proof}
\end{document}