\documentclass[twoside]{article}
\setlength{\oddsidemargin}{0.25 in}
\setlength{\evensidemargin}{-0.25 in}
\setlength{\topmargin}{-0.6 in}
\setlength{\textwidth}{6.5 in}
\setlength{\textheight}{8.5 in}
\setlength{\headsep}{0.75 in}
\setlength{\parindent}{0 in}
\setlength{\parskip}{0.1 in}

%
% ADD PACKAGES here:
%

\usepackage{amsmath,amsfonts,amssymb,graphicx,mathtools,flexisym}

%
% The following commands set up the lecnum (lecture number)
% counter and make various numbering schemes work relative
% to the lecture number.
%
\newcounter{lecnum}
\renewcommand{\thepage}{\thelecnum-\arabic{page}}
\renewcommand{\thesection}{\thelecnum.\arabic{section}}
\renewcommand{\theequation}{\thelecnum.\arabic{equation}}
\renewcommand{\thefigure}{\thelecnum.\arabic{figure}}
\renewcommand{\thetable}{\thelecnum.\arabic{table}}
\newcommand{\N}{\mathbb{N}}
\newcommand{\PS}{\mathcal P}
\newcommand{\Z}{\mathbb{Z}}
\newcommand{\ep}{\varepsilon}
\newcommand{\intr}{\text{int }}


%
% The following macro is used to generate the header.
%

%
% Convention for citations is authors' initials followed by the year.
% For example, to cite a paper by Leighton and Maggs you would type
% \cite{LM89}, and to cite a paper by Strassen you would type \cite{S69}.
% (To avoid bibliography problems, for now we redefine the \cite command.)
% Also commands that create a suitable format for the reference list.
\renewcommand{\cite}[1]{[#1]}
\def\beginrefs{\begin{list}%
        {[\arabic{equation}]}{\usecounter{equation}
         \setlength{\leftmargin}{2.0truecm}\setlength{\labelsep}{0.4truecm}%
         \setlength{\labelwidth}{1.6truecm}}}
\def\endrefs{\end{list}}
\def\bibentry#1{\item[\hbox{[#1]}]}

%Use this command for a figure; it puts a figure in wherever you want it.
%usage: \fig{NUMBER}{SPACE-IN-INCHES}{CAPTION}
\newcommand{\fig}[3]{
			\vspace{#2}
			\begin{center}
			Figure \thelecnum.#1:~#3
			\end{center}
	}
% Use these for theorems, lemmas, proofs, etc.
\newtheorem{theorem}{Theorem}[lecnum]
\newtheorem{lemma}[theorem]{Lemma}
\newtheorem{proposition}[theorem]{Proposition}
\newtheorem{claim}[theorem]{Claim}
\newtheorem{corollary}[theorem]{Corollary}
\newtheorem{definition}[theorem]{Definition}
\newtheorem{example}[theorem]{Example}

\newenvironment{proof}{{\bf Proof:}}{\hfill\rule{2mm}{2mm}}

% **** IF YOU WANT TO DEFINE ADDITIONAL MACROS FOR YOURSELF, PUT THEM HERE:

\newcommand\E{\mathbb{E}}
\newcommand\R{\mathbb{R}}


\begin{document}

\begin{theorem}
    $\exists h > 0$ s.t. IVP/IE has a solution for $t \in [t_0-h, t_0+h]$.
\end{theorem}

\begin{proof}

    (JUST LOOK AT THE NOTES.)

    $f$ is continuous on $A_{h,k}(t_0,x_0)$.

    $\therefore \exists M > 0$ s.t. $|f(t,x)| \leq M \forall (t,x) \in A_{h,k}(t_0, x_0)$. 

    By making $h$ spammer if necessary, we may assume $h \leq \frac k M$.

    $x_n(t) = x + (t - t_0)f(t_0,x_0)$ for $t \in [t_0, t_0 + \frac h n]$. 

    $x_n(t) = x_n(t_0+\frac h n) + [t - (t_0 + \frac h n)]f(t_0 + \frac h n, x^n(t_0+\frac h n)) $ for $t \in [t_0, \frac h n, t_0 + \frac {2h}n]$
    $\vdots$

    $x_n(t) = x_n(t_0+\frac {(i-1)h} n) + [t - (t_0 + \frac {(i-1)h} n)]f(t_0 + \frac {(i-1)h} n, x^n(t_0+\frac {(i-1)h} n)) $ for $t \in [t_0, \frac {(i-1)h} n, t_0 + \frac {{(i)h}}n]$

    The graph of $x_n$ stays inside $A_{h,k}(t_0, x_0)$. Because Lipschitz constant of $x_1 \leq M$. 

    $(x_n)$ is a bounded sequence, Lipschitz constants uniformly bounded by $M$. Therefore $(x_m) \subset $ a compact subset of $C[t_0-h,t_0+h]$ equipped with the usual uniform metric.  

    $\therefore \exists $ subsequence $(x_n')$ s.t. $x_{n'} \rightarrow x \in C[t_0 - h, t_0+h]$ by Arzela-Ascoli theorem.

    WTS $X$ solves IE. 

    $x(t) \leftarrow x_{n'}(t) = x_0 + \int_{t_0}^{t} f(Pn'(s)) ds = x_0 + \int_{t_0}^t Pn'(s) ds  = x_0 + \int_{t_0}^t f(s,x(s)) ds + \int_{t_0}^t [f(P_n'(s)) - f(s,x_{n'}(s))] + \int_{t_0}^t [f(s,x_{n'}(s)) - f(s,x(s))] ds$. 

    Let $t \in [t_0 + \frac{(i-1)h}{n}, t_0 + \frac {ih}{n}]$

    $|(t,x_n(t)) - P_n(t)| \leq \sqrt{ (\frac h n)^2 + M^2(\frac{h^2}{n^2}) } \leq \sqrt{1+M^2}\frac h n$ 
    
    $\frac h n \rightarrow 0 $ as $n \rightarrow \infty$. 

    $\therefore |(t, x_n(t)) - P_n(t)| \rightarrow 0$ uniformly on $[t_0-h, t_0+h]$ as $h \rightarrow \infty$

    Let $\ep > 0$. $f$ is uniformly cts on $A_{h,k}(t_0, x_0)$ $\exists \delta > 0 $ s.t. $P, Q \in A_{h,k}(t_0, x_0) , |P-Q| \leq  \delta$

    $\implies f(P) - f(Q) \leq \ep$.

    $\exists N$ s.t. $n > N \implies |f(t, x_n(t)) - f(P_n(t))| < \ep$ $\forall t \in [t_0 - h, t_0 + h]$

    $\exists \overline N s.t. n' \geq N' \implies |x_n'(x) - x(s)| < \delta \forall s \in [t_0-h, t_0+h]$

\end{proof}

\end{document}