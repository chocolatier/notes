\documentclass[twoside]{article}
\setlength{\oddsidemargin}{0.25 in}
\setlength{\evensidemargin}{-0.25 in}
\setlength{\topmargin}{-0.6 in}
\setlength{\textwidth}{6.5 in}
\setlength{\textheight}{8.5 in}
\setlength{\headsep}{0.75 in}
\setlength{\parindent}{0 in}
\setlength{\parskip}{0.1 in}

%
% ADD PACKAGES here:
%

\usepackage{amsmath,amsfonts,amssymb,graphicx,mathtools,flexisym}

%
% The following commands set up the lecnum (lecture number)
% counter and make various numbering schemes work relative
% to the lecture number.
%
\newcounter{lecnum}
\renewcommand{\thepage}{\thelecnum-\arabic{page}}
\renewcommand{\thesection}{\thelecnum.\arabic{section}}
\renewcommand{\theequation}{\thelecnum.\arabic{equation}}
\renewcommand{\thefigure}{\thelecnum.\arabic{figure}}
\renewcommand{\thetable}{\thelecnum.\arabic{table}}
\newcommand{\N}{\mathbb{N}}
\newcommand{\PS}{\mathcal P}
\newcommand{\Z}{\mathbb{Z}}
\newcommand{\Q}{\mathbb{Q}}
\newcommand{\ep}{\varepsilon}
\newcommand{\intr}{\text{int }}


%
% The following macro is used to generate the header.
%

%
% Convention for citations is authors' initials followed by the year.
% For example, to cite a paper by Leighton and Maggs you would type
% \cite{LM89}, and to cite a paper by Strassen you would type \cite{S69}.
% (To avoid bibliography problems, for now we redefine the \cite command.)
% Also commands that create a suitable format for the reference list.
\renewcommand{\cite}[1]{[#1]}
\def\beginrefs{\begin{list}%
        {[\arabic{equation}]}{\usecounter{equation}
         \setlength{\leftmargin}{2.0truecm}\setlength{\labelsep}{0.4truecm}%
         \setlength{\labelwidth}{1.6truecm}}}
\def\endrefs{\end{list}}
\def\bibentry#1{\item[\hbox{[#1]}]}

%Use this command for a figure; it puts a figure in wherever you want it.
%usage: \fig{NUMBER}{SPACE-IN-INCHES}{CAPTION}
\newcommand{\fig}[3]{
			\vspace{#2}
			\begin{center}
			Figure \thelecnum.#1:~#3
			\end{center}
	}
% Use these for theorems, lemmas, proofs, etc.
\newtheorem{theorem}{Theorem}[lecnum]
\newtheorem{lemma}[theorem]{Lemma}
\newtheorem{proposition}[theorem]{Proposition}
\newtheorem{claim}[theorem]{Claim}
\newtheorem{corollary}[theorem]{Corollary}
\newtheorem{definition}[theorem]{Definition}
\newtheorem{example}[theorem]{Example}

\newenvironment{proof}{{\bf Proof:}}{\hfill\rule{2mm}{2mm}}

% **** IF YOU WANT TO DEFINE ADDITIONAL MACROS FOR YOURSELF, PUT THEM HERE:

\newcommand\E{\mathbb{E}}
\newcommand\R{\mathbb{R}}


\begin{document}

\begin{definition}
    A metric space $(X,d)$ is connected if there do not exist non-empty open sets $U,V$ such that $U \cap V = \emptyset$ and $X = U \cup V$. 

    $X$ is disconnected if it is not connected. 
\end{definition}


If $A \subset X$ is open, then $A \neq X$, then $A \cup A^c = X$

$A^c$ is not open. 

\begin{proposition}
    A metric space $X$ is connected, if and only if there do not exist non-empty disjoint closed sets $E,F$ s.t. $X = E \cup F$. 
\end{proposition}


\begin{proof}
    If $X = U \cup V$, $U, V \neq \emptyset, U \cap V = \emptyset$

    $V = U^c$ closed. 

    $U = V^c$ closed. 
\end{proof}

\begin{proposition}
    A metric space $X$ is connected, if and only if the only subsets of $X$ that are both open and closed are $\emptyset, X$. 
\end{proposition}

\begin{example}
    $\Q = \Q \cap (-\infty, \pi) \cup \Q \cap (\pi, \infty)$

    $[0,1] \cup sin(\frac1x) \subset \R^2$

    \begin{proof}
        Suppose $\exists$ open $U, V$, $U \cap V = \emptyset$, $X =( X\cap U) \cup (X\cap V)$

        Two cases:

        If $X_1 \subset U$, then $X_2 \cap U \neq \emptyset$. $X_2 \cap V \neq \emptyset \implies $ red set is disconnected.

        If $X_1 \not\subset U, X_1 \cap U \neq \emptyset$, then $U, V$ disconnect $X_1$. 
    \end{proof}

\end{example}

\begin{definition}
    A metric space $(X,d)$ is path connected if $\forall a, b \in X$ there exists a continuous function $f:[0,1]\rightarrow X$ s.t. $f(0) = a, f(1) = b$
\end{definition}

Connected $\not\implies$ path connected. 

Connectedness in $\R$. 
\begin{definition}
    $A$ set $S \subset \R$ is on interval if $a,b \in S,  a < c < b \implies c \in S$
\end{definition}

\begin{theorem}
    $S \subset \R$ is connected if and only if it is an interval.
\end{theorem}

\begin{proof}
    Suppose $S$ is not an interval. Then $\exists a, b$ Then $\exists a, b \in S$, say $a < b$ and $x$ s.t. $a < x < b$, $x \not \in S$. 

    Then $S = (S \cap (-\infty, x)) \cup (S \cap (x, \infty))$. 

    $\therefore S$ is not connected. 

    Suppose $S$ is an interval. If $S$ is not connected, then $\exists$ open $U,V \subset S, U\cap V = \emptyset$.

    $U,V \neq \emptyset$. 

    Let $a \in U$, $b \in B, a < b$. 

    Let $c = \sup[a,b]\cap U \subset S$.
    
    Then $c \in U$ or $c \in V$. 

    If $c \in U$, then $c + \ep \in U$ for small enough $\ep > 0$, because $U$ is open. Contradiction because $c = \sup[a,b] \cap U$.

    So $c \in V$. $V$ is open, so $C - \ep \in V$ for small enough $\ep > 0$. Contradiction of $c$ being supremum. 

    $\therefore$ no such $U,V$ exist. 
    
    $\therefore$ S is connected. 
\end{proof}


\begin{theorem}
    In a metric space, path connected $\implies$ connected. 
\end{theorem}

\begin{proof}
    Suppose $X$ is path connected but not connected. 

    Let $a, b \in X$, $f:[0,1] \rightarrow X$ continuous. 

    $f(0) = a$

    $f(1) = b$.

    If $X = U \cup V$ open. 

    $U \cap V = \emptyset$

    $U,V \neq \emptyset$.

    Then $f^{-1}(U), f^{-1}(V)$ give a disconnection of $[0,1]$. Contradiction. 
    
    $\therefore X$ is connected. 
\end{proof}
\end{document}