\documentclass[twoside]{article}
\setlength{\oddsidemargin}{0.25 in}
\setlength{\evensidemargin}{-0.25 in}
\setlength{\topmargin}{-0.6 in}
\setlength{\textwidth}{6.5 in}
\setlength{\textheight}{8.5 in}
\setlength{\headsep}{0.75 in}
\setlength{\parindent}{0 in}
\setlength{\parskip}{0.1 in}

%
% ADD PACKAGES here:
%

\usepackage{amsmath,amsfonts,amssymb,graphicx,mathtools,flexisym}

%
% The following commands set up the lecnum (lecture number)
% counter and make various numbering schemes work relative
% to the lecture number.
%
\newcounter{lecnum}
\renewcommand{\thepage}{\thelecnum-\arabic{page}}
\renewcommand{\thesection}{\thelecnum.\arabic{section}}
\renewcommand{\theequation}{\thelecnum.\arabic{equation}}
\renewcommand{\thefigure}{\thelecnum.\arabic{figure}}
\renewcommand{\thetable}{\thelecnum.\arabic{table}}
\newcommand{\N}{\mathbb{N}}
\newcommand{\PS}{\mathcal P}
\newcommand{\Z}{\mathbb{Z}}
\newcommand{\Q}{\mathbb{Q}}
\newcommand{\ep}{\varepsilon}
\newcommand{\intr}{\text{int }}


%
% The following macro is used to generate the header.
%

%
% Convention for citations is authors' initials followed by the year.
% For example, to cite a paper by Leighton and Maggs you would type
% \cite{LM89}, and to cite a paper by Strassen you would type \cite{S69}.
% (To avoid bibliography problems, for now we redefine the \cite command.)
% Also commands that create a suitable format for the reference list.
\renewcommand{\cite}[1]{[#1]}
\def\beginrefs{\begin{list}%
        {[\arabic{equation}]}{\usecounter{equation}
         \setlength{\leftmargin}{2.0truecm}\setlength{\labelsep}{0.4truecm}%
         \setlength{\labelwidth}{1.6truecm}}}
\def\endrefs{\end{list}}
\def\bibentry#1{\item[\hbox{[#1]}]}

%Use this command for a figure; it puts a figure in wherever you want it.
%usage: \fig{NUMBER}{SPACE-IN-INCHES}{CAPTION}
\newcommand{\fig}[3]{
			\vspace{#2}
			\begin{center}
			Figure \thelecnum.#1:~#3
			\end{center}
	}
% Use these for theorems, lemmas, proofs, etc.
\newtheorem{theorem}{Theorem}[lecnum]
\newtheorem{lemma}[theorem]{Lemma}
\newtheorem{proposition}[theorem]{Proposition}
\newtheorem{claim}[theorem]{Claim}
\newtheorem{corollary}[theorem]{Corollary}
\newtheorem{definition}[theorem]{Definition}
\newtheorem{example}[theorem]{Example}

\newenvironment{proof}{{\bf Proof:}}{\hfill\rule{2mm}{2mm}}

% **** IF YOU WANT TO DEFINE ADDITIONAL MACROS FOR YOURSELF, PUT THEM HERE:

\newcommand\E{\mathbb{E}}
\newcommand\R{\mathbb{R}}


\begin{document}
\section{Connected Spaces}
    $X$ is connected if there do not exist disjoint open sets $U, V \neq \emptyset$ s.t. $X = U \cup V$. 

    $X$ is path connected if for all $x,y \in X$ there exists a continuous function $f:[0,1] \rightarrow X$ s.t. $f(0) = x$ and $f(1) = y$. 

    \begin{theorem}
        Path connected $\implies$ connected, but not vice-versa.  
    \end{theorem}

    \begin{theorem}
        Suppose $U$ is an open set in $\R^n$. Then $U$ is connected, if and only if it is path connected. 
    \end{theorem}

    \begin{proof}
        Suppose $U$ is connected. 

        WTS $U$ is path connected. 

        Let $U \neq \emptyset$. Let $a \in U$. 
        
        $E = \{x\in U \vert \exists\text{a path in $U$ from $a$ to $x$}\}$.

        Claim: $E$ is both open and closed (in $U$). 

        Let $x \in E$. Then $\exists$ a path from $a$ to $x$. 

        $B_r(x) \subset U$. 

        $g(t)  = (1-t)x+tz, t \in [0,1]$. 

        Then $\exists $ a path from $a$ to any $z \in B_r$, by following the path from $a to x$ and the radial path from $x$ to $z$. 

        $\therefore E$ is open. 

        E is closed. 

        Let $(x_n) \subset E$, $x_n \rightarrow x \in U$. WTS $x \in E$. 

        $x \in U \implies \exists r > 0$ s.t. $B_r(x) \subset U$. 

        We can connect every point, so once we are in $B_r(x)$, take radial path. 

        $E$ is both open and closed in $U$.

        $\therefore E = U$ because $U$ is connected and $E \neq \emptyset$
    \end{proof}

    \begin{theorem}
        The continuous image of a connected set is connected.
    \end{theorem}

    \begin{proof}
        Let $X$ be connected, $f: X \rightarrow Y$ continuous. 

        Suppose $f(X)$ is not connected. 

        Then $\exists$ open sets $U, V \subset Y$ s.t. $U \cap V = \emptyset$ and $U \cap f(X)$, $V \cap f(X)$.

        $U \cup V = Y$. 

        $f^{-1}(U)$ is non-empty, $f^{-1}(V)$ is non-empty.
        
        $f^{-1}(U) \cap f^{-1}(V)$. 

        $f^{-1}(U),f^{-1}(V)$ open.

        $f^{-1}(U) \cup f^{-1}(V) = X \implies X$ is not connected.  
    \end{proof}

    \begin{corollary}
        If $X$ is connected, $f: X \rightarrow \R$ is continuous. 

        Let $x, y \in X$, $f(x) =a$

        $f(y) = b$. 

        Suppose $a < b$.

        Then for any $c \in (a,b) \exists z \in X$ s.t. $f(z) = c$. 
    \end{corollary}

    \begin{proof}
        $f(X)$ is connect, $\therefore$ it is an interval. 
    \end{proof}

    \section{Inverse Function Theorem}

    Suppose $f: I \rightarrow \R$, $I$ an open interval. 

    $f$ is differentiable at $t_0 \in I$ if $\underset{t\rightarrow t_0}\lim\frac{f(t) - f(t_0)}{t-t_0}$. 

    Equivalently $lim_{n\rightarrow \infty} \frac{f(t_0 + h) - f(t_0)}{h}$ exists. 

    Write $f'(t_0)$ for this limit if it exists.
    
    $f'(t_0) = \lim_{h \rightarrow 0}\frac{f(t_0 + h) - f(t_0)}{h}$

    $f'(t_0) = \frac{f(t_0 + h) - f(t_0)}{h} + E(h)$

    Then $\frac{f(t_0 + h) - [f(t_0) + f'(t_0)h]}{h} + E(h) = 0$

    Then $\lim_{h\rightarrow \infty} \frac{f(t_0 + h) - [f(t_0) + f'(t_0)h]}{h} = 0$

    If $f(t_0 + h) - [f(t_0) + f'(t_0)h] = \tilde E(h)$, where $\tilde E(h) \rightarrow 0$ faster than $h$. 

    $\frac{\tilde E(h)}{h}\rightarrow 0$ as $h \rightarrow \infty$. 

    Let $f: U \rightarrow \R$. U open in $\R^n$. 

    Let $x_0 \in U$. $f$ is differentiable at $x_0$ if $\exists$ a linear fcn $L: \R^n \rightarrow \R$ s.t. $f(x_0 + h) - [f(x_0) + Lh] = \tilde E(h)$ where $\tilde E(h) \rightarrow 0$ faster than $|h|$, $|h| \in \R^n$. 

    $e_i = (0,\dots,1,\dots, \dots,0)$

    $\lim_{t \rightarrow 0}\frac{f(x_0+t_0i) - f(x_0))}{t} = \frac{\partial f}{\partial x}(x_0)$. 

    $D_nf(x_0) = \lim_{t \rightarrow 0}\frac{f(x_0+t_v) - f(x_0))}{t}$.

    $L = [\frac{\partial f}{\partial x_1}(x_0), \dots, \frac{\partial f}{\partial x_n}(x_n)]$
\end{document}