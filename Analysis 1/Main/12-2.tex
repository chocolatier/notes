\documentclass[twoside]{article}
\setlength{\oddsidemargin}{0.25 in}
\setlength{\evensidemargin}{-0.25 in}
\setlength{\topmargin}{-0.6 in}
\setlength{\textwidth}{6.5 in}
\setlength{\textheight}{8.5 in}
\setlength{\headsep}{0.75 in}
\setlength{\parindent}{0 in}
\setlength{\parskip}{0.1 in}

%
% ADD PACKAGES here:
%

\usepackage{amsmath,amsfonts,amssymb,graphicx,mathtools,flexisym}

%
% The following commands set up the lecnum (lecture number)
% counter and make various numbering schemes work relative
% to the lecture number.
%
\newcounter{lecnum}
\renewcommand{\thepage}{\thelecnum-\arabic{page}}
\renewcommand{\thesection}{\thelecnum.\arabic{section}}
\renewcommand{\theequation}{\thelecnum.\arabic{equation}}
\renewcommand{\thefigure}{\thelecnum.\arabic{figure}}
\renewcommand{\thetable}{\thelecnum.\arabic{table}}
\newcommand{\N}{\mathbb{N}}
\newcommand{\PS}{\mathcal P}
\newcommand{\Z}{\mathbb{Z}}
\newcommand{\Q}{\mathbb{Q}}
\newcommand{\ep}{\varepsilon}
\newcommand{\intr}{\text{int }}


%
% The following macro is used to generate the header.
%

%
% Convention for citations is authors' initials followed by the year.
% For example, to cite a paper by Leighton and Maggs you would type
% \cite{LM89}, and to cite a paper by Strassen you would type \cite{S69}.
% (To avoid bibliography problems, for now we redefine the \cite command.)
% Also commands that create a suitable format for the reference list.
\renewcommand{\cite}[1]{[#1]}
\def\beginrefs{\begin{list}%
        {[\arabic{equation}]}{\usecounter{equation}
         \setlength{\leftmargin}{2.0truecm}\setlength{\labelsep}{0.4truecm}%
         \setlength{\labelwidth}{1.6truecm}}}
\def\endrefs{\end{list}}
\def\bibentry#1{\item[\hbox{[#1]}]}

%Use this command for a figure; it puts a figure in wherever you want it.
%usage: \fig{NUMBER}{SPACE-IN-INCHES}{CAPTION}
\newcommand{\fig}[3]{
			\vspace{#2}
			\begin{center}
			Figure \thelecnum.#1:~#3
			\end{center}
	}
% Use these for theorems, lemmas, proofs, etc.
\newtheorem{theorem}{Theorem}[lecnum]
\newtheorem{lemma}[theorem]{Lemma}
\newtheorem{proposition}[theorem]{Proposition}
\newtheorem{claim}[theorem]{Claim}
\newtheorem{corollary}[theorem]{Corollary}
\newtheorem{definition}[theorem]{Definition}
\newtheorem{example}[theorem]{Example}

\newenvironment{proof}{{\bf Proof:}}{\hfill\rule{2mm}{2mm}}

% **** IF YOU WANT TO DEFINE ADDITIONAL MACROS FOR YOURSELF, PUT THEM HERE:

\newcommand\E{\mathbb{E}}
\newcommand\R{\mathbb{R}}


\begin{document}

    \begin{definition}
        $f: U \subset \R^n \rightarrow \R$, $U$ open is differentiable at $a \in U$ if $\exists$ a linear map $L:\R^n \rightarrow \R$ s.t. $f(a+h) - (f(a) + f(h)) + E(h) = 0$, for $h \in \R^n$, $(h)$ small, $\frac{E(h)}{|h|} \rightarrow 0$ as $h \rightarrow 0$. 

        L is unique if it exists. 

        $L = (\frac{\partial f}{\partial x_1}(a), \dots, \frac{\partial f}{\partial x_n}(a))$

        Equivalently, 

        $\frac{f(a+h) - (f(a) + L h)}{|h|}$.

        If $f: U \subset \R^n \rightarrow \R^m$. 

        $f$ is differentiable if every component is differentiable, 

        $f = (f_1, \dots, f_m)^T$. 

        Equivalently, $f$ is differentiable if $\exists$ a linear map $L:\R^n \rightarrow \R^m$ s.t. $f(a+h) - (f(0) + Lh) + E(h) = 0$, where $\frac{E(h)}{|h|} \rightarrow 0$
    \end{definition}


    \begin{theorem}
        (Inverse Function Theorem)

        Suppose $f:U\subset \R^n \rightarrow \R^n$. 

        $(i.e. frac{\partial f}{\partial x_1}, \dots, \frac{\partial f}{\partial x_n})$ are continuous in $U$
    
        Let $a \in U$, and suppose $df(a): \R^n \rightarrow \R^n$ is invertible. 

        Then $\exists$ open sets containing a and $W$ containing $f(a)$, s.t. $f\vert_V:V\rightarrow W$ is invertible. 

        $f^{-1}: W \rightarrow V$ is also $C^1$ and $d(f^{-1})(y) = [df(f^{-1}(y))]^{-1}$.

        If $f \in C^k(U)$ for some $k \in \N$ then $f^{-1} \in C^k(W)$. 

    \end{theorem}

    \begin{proof}
        By replacing $f$ by $[df(a)]^{-1} \circ f$, we can assume $df(a) = I$. 

        $d[df(a)^{-1}\circ f](a) = df(a)^{-1} \cdot df(a)=I.$

        Let $x_1, x_2 \in B_r(a) \subset U$

        $f(x_1) - f(x_2) = df(z) \cdot (x_1 -x_2)$ for some $z \in [x_1, x_2]$. 

        $(x_1 - x_2) + (d(f(z) - I)(x_1 - x_2)$

        $\therefore $ for small enough $r$, we have $\frac12|x_1 - x_2| < |f(x_1) - f(x_2)| \leq 2|x_1 - x_2|$, for all $x_1, x_2 \in B_r(0)$.

        $f$ is one to one on $\overline{B_r(a)}$.

        $f\vert_{\overline {B_r(a)}}$ has an inverse $f^{-1}$ defined on $f(\overline {B_r(a)})$.

        $|f^{-1}(y_1) - f^{-1}(y_2)| \leq 2 |y_1 - y_2|$. 

        Next step: Show $f(B_r(a))$ contains an open neighbourhood of $f(a)$. 

        $f$ is one to one on $\overline{B_r(a)}$. 

        $\therefore f(a) \not\in f(\partial B_r(a))$. 

        $f(\partial B_r(a))$ is compact. 

        $\therefore \exists \delta > 0$ such that $|f(x) - f(a)| > \delta \forall x \in \delta B_r(a)$

        Let $W = B_{S/2}(f(a))$.
        
        If $y \in W, x \in \partial B_r(a)$. 

        $|f(x) - y| = |f(x) - f(a) + f(a) - y| \geq |f(x) - f(a)| - |f(a) - y| \geq \delta - \frac\delta 2 = \frac{\delta}{2} > |f(a) - y|$. 

        Let $g(x) = |f(x) - y|^2$. 

        $g(x) > g(a)$ for all x $\in B_r(a)$. 

        $\therefore$ g must achieve it's minimum at an interior point. 

        At that point, $\frac{\partial g}{\partial x_i} = 2 \sum_{i=1}^n (f^{i} - y^i)\frac{\partial f^i}{\partial  x_j}$. 

        In matrix form $0 = [\frac{\partial f^i}{\partial  x_j}](f'(x) - y', \dots, f^n(x) - y^n)^T$.

        $\implies f(x) = Y$. 

        This shows for every $y \in W$, $\exists x \in B_r(a)$ s.t. $f(x) = y$.

        Define $V = f^{-1(W)} \cap B_r(a)$. 

        Let $x_1, x_2 \in V$. $y_1 = f(x_1)$, $y_2 = f(x_2) \in W$. 

        $f(x_1) - f(x_2) \subset df(x_2)\times(x_1-x_2) + E(x_1 - x_2)$. Multiply both sides by $[df(x_2)]^{-1}$.

        $[df(x_2)]^{-1} \rightarrow 0 = f^{-1} - f^{-1}(y_2) + [df(x_2)]^{-1}E(x_1 - x_2)$. 

        $df^{-1}(y_2) = [df(f^{-1}(y_2))]^{-1}$. 

        $[df(x_2)]^{-1}$ is bounded. 

        $\frac{E(x_1 - x_2)}{|x_1 - x_2|} \times \frac{|x_1 - x_2|}{|y_1 - y_2|} \rightarrow 0$ as $y_1 \rightarrow y$. 

        $\frac13|x_1 - x_2| \leq |y_1 - y_2| \leq 2|x_1 - x_2|$

        $\implies f^{-1}$ is differentiable at $x_2$. 
    \end{proof}

\end{document}