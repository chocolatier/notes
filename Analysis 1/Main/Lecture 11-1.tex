\documentclass[twoside]{article}
\setlength{\oddsidemargin}{0.25 in}
\setlength{\evensidemargin}{-0.25 in}
\setlength{\topmargin}{-0.6 in}
\setlength{\textwidth}{6.5 in}
\setlength{\textheight}{8.5 in}
\setlength{\headsep}{0.75 in}
\setlength{\parindent}{0 in}
\setlength{\parskip}{0.1 in}

%
% ADD PACKAGES here:
%

\usepackage{amsmath,amsfonts,amssymb,graphicx,mathtools,flexisym}

%
% The following commands set up the lecnum (lecture number)
% counter and make various numbering schemes work relative
% to the lecture number.
%
\newcounter{lecnum}
\renewcommand{\thepage}{\thelecnum-\arabic{page}}
\renewcommand{\thesection}{\thelecnum.\arabic{section}}
\renewcommand{\theequation}{\thelecnum.\arabic{equation}}
\renewcommand{\thefigure}{\thelecnum.\arabic{figure}}
\renewcommand{\thetable}{\thelecnum.\arabic{table}}
\newcommand{\N}{\mathbb{N}}
\newcommand{\PS}{\mathcal P}
\newcommand{\Z}{\mathbb{Z}}
\newcommand{\intr}{\text{int }}
\newcommand{\FF}{\mathcal{F}}
\renewcommand{\d}{\delta}
\renewcommand{\a}{\alpha}
\newcommand{\ep}{\varepsilon}


%
% The following macro is used to generate the header.
%

%
% Convention for citations is authors' initials followed by the year.
% For example, to cite a paper by Leighton and Maggs you would type
% \cite{LM89}, and to cite a paper by Strassen you would type \cite{S69}.
% (To avoid bibliography problems, for now we redefine the \cite command.)
% Also commands that create a suitable format for the reference list.
\renewcommand{\cite}[1]{[#1]}
\def\beginrefs{\begin{list}%
        {[\arabic{equation}]}{\usecounter{equation}
         \setlength{\leftmargin}{2.0truecm}\setlength{\labelsep}{0.4truecm}%
         \setlength{\labelwidth}{1.6truecm}}}
\def\endrefs{\end{list}}
\def\bibentry#1{\item[\hbox{[#1]}]}

%Use this command for a figure; it puts a figure in wherever you want it.
%usage: \fig{NUMBER}{SPACE-IN-INCHES}{CAPTION}
\newcommand{\fig}[3]{
			\vspace{#2}
			\begin{center}
			Figure \thelecnum.#1:~#3
			\end{center}
	}
% Use these for theorems, lemmas, proofs, etc.
\newtheorem{theorem}{Theorem}[lecnum]
\newtheorem{lemma}[theorem]{Lemma}
\newtheorem{proposition}[theorem]{Proposition}
\newtheorem{claim}[theorem]{Claim}
\newtheorem{corollary}[theorem]{Corollary}
\newtheorem{definition}[theorem]{Definition}
\newtheorem{example}[theorem]{Example}

\newenvironment{proof}{{\bf Proof:}}{\hfill\rule{2mm}{2mm}}

% **** IF YOU WANT TO DEFINE ADDITIONAL MACROS FOR YOURSELF, PUT THEM HERE:

\newcommand\E{\mathbb{E}}
\newcommand\R{\mathbb{R}}


\begin{document}
\begin{proof} {Arzela-Ascoli Theorem}
    
    We need to show $\mathcal F$ is complete and totally bounded.

    $(C(X, \R^n), d_n)$ is complete, $\FF$ is closed in $C(X,\R^n) \implies \FF $ is complete

    Let $\d > 0$. WTS $\exists$ a finite collection $S$ of functions in $C(X, \R^m)$ s.t. $\forall f \in \FF$, $\exists g \in S \text{s.t.} d_u(f,g) < \d$.

    $\FF $ is bounded, so there exists $K$, s.t.  $|f(x)| \leq K$, $\forall x \in X$, $\forall f \in \FF$.

    By uniform continuity, $\exists \d > 0$ s.t. $d(u,v) < \d_1 \implies |f(u) - f(v)| < \frac\ep4$, $\forall x \in X$, $\forall f \in \FF$.

    By total boundedness of $X$, $\exists x_1, \dots, x_p \in X$ s.t. for some $i \in 1,\dots,p$ then $|f(x) - f(x_i)| < \frac\d4 \forall f \in \FF$

    By total boundedness of $B_k(0) \subset \R^n$, $\exists$ finitely many $y_1, y_2, \dots, y_q \in \R^n$ s.t. $|y - y_j| \leq \frac\d4$

    Consider all functions $\alpha: \{x_1, \dots, x_p\} \rightarrow \{y_1, \dots, y_p\}$.

    There are finitely many such $\alpha$, $q^p$.

    For each $\alpha$, if $\exists f \in \FF$ such that $|f(x_i) - \a(x_i)| < \frac\d4, \forall i = 1, \dots, p$

    Choose one such $f$, call it $g_\alpha$. 

    Let $S = \{g_\alpha\}$. This is a finite collection. 

    Then $|g_\alpha(x_i) - \a(x_i)| < \frac\d4 \forall i = 1, \dots, p $

    Let $f \in \FF$. For each $i = 1, \dots, p$ choose one $y_j$ s.t. $|f(x_i) - y_j| < \frac\d4$.

    Let $\a$ be a function that assigns to each $x_i$ the corresponding $y_j$. Then $|f(x_i) - \a(x_i)| < \frac\d4 \forall i = 1, \dots, p$

    Then $\exists g_\a$ as above. 

    Let $x \in X$. Choose $x_i$ such that $d(x, x_i) < \d_1$. 

    $|f(x) - g_\a(x)| \leq |f(x) - f(x_i)| + |f(x_i) - \a(x_i)| + |\a(x_i) - g_\a(x_i)| + |g_\a(x_i) - g_\a(x)| <  \frac\d4 + \frac\d4 + \frac\d4 + \frac\d4 = \d$

\end{proof}


    $\circ$ $x'(t) = f(t, x(t))$, $f$ cts in $U \subset \R \times \R$. 

    If in addition f is locally Lipschitz cts with respect to x, then $\exists h > 0 $ such that $(*)$ has a unique solution defined for $t \in [t_0 - h, t_0 + h]$.

    If $f$ is not locally Lipschitz, then uniqueness can fail. 

    E.g. $x'(t) = \sqrt{|x(t)|}$

    $x(0) = 0$

    Existence still holds for small enough $h$. 

\begin{theorem} [Peano's Theorem]

    Suppose $f \in C(U)$, $(t_0, x_0) \in U$

    Then $\exists h > 0$ s.t. $(*)$ has a solution on $[t_0-h,t_0+h]$


\end{theorem}


$\circ \iff $

$\circ\circ$ $x(t) = x_0 + \int_{t_0}^t f(s, x(s)) ds$

$x(t) = x_0 + f(t_0, x_0)(t-t0)$

$x(t) = x(t_1) + f(t_1, x(t_1))(t-t_1)$

Approximate w piecewise linear fcns, that limits to the function. 

\end{document}