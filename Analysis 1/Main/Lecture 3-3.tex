\documentclass[twoside]{article}
\setlength{\oddsidemargin}{0.25 in}
\setlength{\evensidemargin}{-0.25 in}
\setlength{\topmargin}{-0.6 in}
\setlength{\textwidth}{6.5 in}
\setlength{\textheight}{8.5 in}
\setlength{\headsep}{0.75 in}
\setlength{\parindent}{0 in}
\setlength{\parskip}{0.1 in}

%
% ADD PACKAGES here:
%

\usepackage{amsmath,amsfonts,amssymb,graphicx,mathtools,flexisym}

%
% The following commands set up the lecnum (lecture number)
% counter and make various numbering schemes work relative
% to the lecture number.
%
\newcounter{lecnum}
\renewcommand{\thepage}{\thelecnum-\arabic{page}}
\renewcommand{\thesection}{\thelecnum.\arabic{section}}
\renewcommand{\theequation}{\thelecnum.\arabic{equation}}
\renewcommand{\thefigure}{\thelecnum.\arabic{figure}}
\renewcommand{\thetable}{\thelecnum.\arabic{table}}
\newcommand{\N}{\mathbb{N}}
\newcommand{\PS}{\mathcal P}
\newcommand{\Z}{\mathbb{Z}}


%
% The following macro is used to generate the header.
%

%
% Convention for citations is authors' initials followed by the year.
% For example, to cite a paper by Leighton and Maggs you would type
% \cite{LM89}, and to cite a paper by Strassen you would type \cite{S69}.
% (To avoid bibliography problems, for now we redefine the \cite command.)
% Also commands that create a suitable format for the reference list.
\renewcommand{\cite}[1]{[#1]}
\def\beginrefs{\begin{list}%
        {[\arabic{equation}]}{\usecounter{equation}
         \setlength{\leftmargin}{2.0truecm}\setlength{\labelsep}{0.4truecm}%
         \setlength{\labelwidth}{1.6truecm}}}
\def\endrefs{\end{list}}
\def\bibentry#1{\item[\hbox{[#1]}]}

%Use this command for a figure; it puts a figure in wherever you want it.
%usage: \fig{NUMBER}{SPACE-IN-INCHES}{CAPTION}
\newcommand{\fig}[3]{
			\vspace{#2}
			\begin{center}
			Figure \thelecnum.#1:~#3
			\end{center}
	}
% Use these for theorems, lemmas, proofs, etc.
\newtheorem{theorem}{Theorem}[lecnum]
\newtheorem{lemma}[theorem]{Lemma}
\newtheorem{proposition}[theorem]{Proposition}
\newtheorem{claim}[theorem]{Claim}
\newtheorem{corollary}[theorem]{Corollary}
\newtheorem{definition}[theorem]{Definition}
\newtheorem{example}[theorem]{Example}

\newenvironment{proof}{{\bf Proof:}}{\hfill\rule{2mm}{2mm}}

% **** IF YOU WANT TO DEFINE ADDITIONAL MACROS FOR YOURSELF, PUT THEM HERE:

\newcommand\E{\mathbb{E}}
\newcommand\R{\mathbb{R}}


\begin{document}
%FILL IN THE RIGHT INFO.
%\lecture{**LECTURE-NUMBER**}{**DATE**}{**LECTURER**}{**SCRIBE**}
% \lecture{2}{March 03}{Analysis Extension}{Axiomatic Set Theory}
%\footnotetext{These notes are partially based on those of Nigel Mansell.}

% **** YOUR NOTES GO HERE:

\section{Metric Spaces}

\begin{definition}
    A Metric space $(X,d)$ is a set $X$ together with a function $d : X \times X \rightarrow \R$ s.t.

    \begin{itemize}
        \item Positivity: $d(x,y) \geq 0$, $= 0 \iff x = y$
        \item Symmetry: $d(y,x) = d(x,y)$
        \item Triangle inequality: $d(x,y) \leq d(x,z) + d(z,y)$
    \end{itemize}
\end{definition}

\begin{example}
    \begin{itemize}
        \item $\R, d(x,y) = |x-y|$
        \item $R^n, d(x,y) = \big( \sum_{i+1}^n |x_i - y_i|^2\big)^{\frac12}$

        $d_p(x,y) = \big( \sum_{i+1}^n |x_i - y_i|^p\big)^{\frac1p}$
        
        $p = 1$ is the Manhattan Metric
        
        $d_{\infty} (x,y) = max_{1\leq i \leq n} |x_i - y_i|$
        \item Any normed space $(V, \|\cdot\|)$
        \item X any set $(\neq \emptyset)$


        $d(x,y) = 1 \text{ if } x \neq y, 0 \text{ if } x = y$

        \item TODO: Insert diagram Globe

        \item TODO: Insert diagram Toruses

        \item Paris metro or post office metric 

        $d(x,y) = \|x-y\| $ if $y = tx $ for some $t \in \R$

        $|x| + |y|$ otherwise 

        \item P-adic Metric: $\Z$, let $p$ be a prime. 

        If $x,y \in \Z$, $x - y = p^kn$, $p $ does not divide $n$

        $d(x,y) = 1/(k+1)$ if $x \neq y$

        $0$ if $x = y$
    \end{itemize}
\end{example}

\begin{definition}
    Let $(X,d)$ be the a metric space. 

    The open ball of radius $r$, centre $a$ is the set $B_r(a) = \{x \in X 
    \vert d(x,a) < r\}$
\end{definition}

\begin{example}
    \item $\R$ usual metric, $B_r = (a-r,a+r)$

    \item $\R^n B_1(0) \in \R^n, d $

    TODO: Insert Diagrams for different metrics

    \item B_r(x) = \{x\} if $r \leq 1$

    B_r(x) = X if $r \geq 1$
\end{example}

\begin{definition}
    $Y \subset X$ is a neighbourhood of a point $x \in X$ if $\exists r > 0$ s.t $B_r(x) \subset Y$
\end{definition}

\begin{definition}
    A set $X \subset X$ is bounded if $S \subset B_r(x)$ for some $x 
    \in X, r > 0$
\end{definition}

\begin{proposition}
    If $S \subset X$ is bounded, then for any $y \in X \exists r > 0$ s.t. $S \subset B_r(y)$
\end{proposition}

\begin{proof}
    Ex: Use triangle ineq
\end{proof}

\begin{definition}
    Suppose $A \subset X$. A point $x \in X$ is an interior point of $A$ if $\exists r>0$ s.t. $B_r(x) \subset A$ 

    $x \in X$ is an exterior point if $A$ if $\exists r > 0$ s.t. $B_r (x) \subset A^c = X - A$

    A point $x \in X$ is a boundary point of $A$ if $\forall r > 0$, if $B_r(x)$ contains points of A and points of $A^c$

    Set of interior points = int A 
    exterior = ext A 
    boundary = $\delta A$

\end{definition}

\begin{proposition}
    $X = int A \cup ext A \cup \delta A$. These sets are pairwise disjoint, 

    $ext A = int(A^c)$

    $int A = ext(A^c)$

    $int A \subset A$, $ext A \subset A^c$
\end{proposition}

\end{document}