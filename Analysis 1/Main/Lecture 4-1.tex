\documentclass[twoside]{article}
\setlength{\oddsidemargin}{0.25 in}
\setlength{\evensidemargin}{-0.25 in}
\setlength{\topmargin}{-0.6 in}
\setlength{\textwidth}{6.5 in}
\setlength{\textheight}{8.5 in}
\setlength{\headsep}{0.75 in}
\setlength{\parindent}{0 in}
\setlength{\parskip}{0.1 in}

%
% ADD PACKAGES here:
%

\usepackage{amsmath,amsfonts,amssymb,graphicx,mathtools,flexisym}

%
% The following commands set up the lecnum (lecture number)
% counter and make various numbering schemes work relative
% to the lecture number.
%
\newcounter{lecnum}
\renewcommand{\thepage}{\thelecnum-\arabic{page}}
\renewcommand{\thesection}{\thelecnum.\arabic{section}}
\renewcommand{\theequation}{\thelecnum.\arabic{equation}}
\renewcommand{\thefigure}{\thelecnum.\arabic{figure}}
\renewcommand{\thetable}{\thelecnum.\arabic{table}}
\newcommand{\N}{\mathbb{N}}
\newcommand{\PS}{\mathcal P}
\newcommand{\Z}{\mathbb{Z}}


%
% The following macro is used to generate the header.
%

%
% Convention for citations is authors' initials followed by the year.
% For example, to cite a paper by Leighton and Maggs you would type
% \cite{LM89}, and to cite a paper by Strassen you would type \cite{S69}.
% (To avoid bibliography problems, for now we redefine the \cite command.)
% Also commands that create a suitable format for the reference list.
\renewcommand{\cite}[1]{[#1]}
\def\beginrefs{\begin{list}%
        {[\arabic{equation}]}{\usecounter{equation}
         \setlength{\leftmargin}{2.0truecm}\setlength{\labelsep}{0.4truecm}%
         \setlength{\labelwidth}{1.6truecm}}}
\def\endrefs{\end{list}}
\def\bibentry#1{\item[\hbox{[#1]}]}

%Use this command for a figure; it puts a figure in wherever you want it.
%usage: \fig{NUMBER}{SPACE-IN-INCHES}{CAPTION}
\newcommand{\fig}[3]{
			\vspace{#2}
			\begin{center}
			Figure \thelecnum.#1:~#3
			\end{center}
	}
% Use these for theorems, lemmas, proofs, etc.
\newtheorem{theorem}{Theorem}[lecnum]
\newtheorem{lemma}[theorem]{Lemma}
\newtheorem{proposition}[theorem]{Proposition}
\newtheorem{claim}[theorem]{Claim}
\newtheorem{corollary}[theorem]{Corollary}
\newtheorem{definition}[theorem]{Definition}
\newtheorem{example}[theorem]{Example}

\newenvironment{proof}{{\bf Proof:}}{\hfill\rule{2mm}{2mm}}

% **** IF YOU WANT TO DEFINE ADDITIONAL MACROS FOR YOURSELF, PUT THEM HERE:

\newcommand\E{\mathbb{E}}
\newcommand\R{\mathbb{R}}


\begin{document}

$(X,d) $ a metric space, $A \subset X$

$x \in A \subset X$ is an interior point of $A$ if $\exists r > 0$ s.t.$B_r(x) \subset A$.

$x \in A \subset X$ is an exterior point if $A$ if $\exists r > 0$ s.t $B_r(x) \subset A^c$

$x \in A \subset X$ is a boundary point of A if every ball $B_r(a)$ contains points of $A$ and $A^c$

Prop. $X = int A \cup \partial A \cup ext A$ (pairwise disjoint)

Ext A = $int A^c$, $int A = ext(A^c)$

$int A \subset A$, $ext A \subset A^c$

TODO: Diagram. 

\begin{definition}
    $x \in X$ is a limit point of A if every ball $B_r(x)$ contains points of A other than X.
    X may or may not be an element of $A$. 

    E.g. (0,1]

    0 is a limit point, 1 is a limit point. 
    $0 \not\in A$, $1 \in A$

    $x \in A$ is an isolated point of A if $\exists r > 0$ s.t. $B_r(x) \cup A = \{x\}$

    Closure of A $\bar A = A \cup \{\text{limit points of a}\}$
\end{definition}

\begin{proposition}

    \begin{enumerate}

        \item A limit point of $L$ of need not be an element of $A$
        \item If x is a limit point of A then every open ball $B_r(x)$ contains infinitely many points of $A$. 
        \item $A \subset \bar A$ 
        \item Every point of $\bar A$ is either a limit point of A or an isolated point of A, but not both. 
        \item $x \in \bar A$ iff every ball $B_r(x)$ contains a point fo $A$.  
    \end{enumerate}   
\end{proposition}

\begin{proof}
   2) If finitely many points, just draw a ball of smaller distance -> isolated point, not limit point by contradiction. 
\end{proof}

Let $A = [0,1] \cup {3} \subset \R$

$int A = (0,1)$
$\partial A = \{ 0, 1, 3\}$

3 is an isolated point 

Limit points of $A = [0,1]$

Let $A = [0,1] \cup {3} \subset \ [0,2] \cup \{3\}$

$int A = [0,1)$

\begin{theorem}
    $\bar A = (ext A)^x$

    $\bar A = int A \cup \partial A$


    $\bar A = A \cup \partial A$
\end{theorem}

\begin{proof}
    1) $x \in \bar A$ iff every ball $B_r$(x) contains at least one element of A. 

    2) WEe have $X = int A \cup \partial A \cup ext A$, but these are pairwise disjoint. It follws that $\bar A = int A \cup \partial A$
    
    3) Let $x \in A \cup \partial A$

    $x \in \partial A \implies x \in int A \cup \partial A = \bar A$

\end{proof}

\subsection{Special case $A = B_r(x) \subset \R^n$} 

$int A = A$

$ext A = \{y : d(y,x) > r\}$

$\partial A = {y : d(y,x) = r}$

$\bar A = \{y : d(y,x) \leq  r\}$ 

\subsection{General case }

$int A = A$

$ext A \supset \{y : d(y,x) > r\}$

$\partial A \subset {y : d(y,x) = r}$

$\bar A \subset \{y : d(y,x) \leq  r\}$ 

Ex. $X = {0,1}$


    d(0,0) = 0
    d(1,1) = 0
    d(0,1) - d(1,0) = 1

A = $B_1(0) = \{x : d(x,0) < 1\} = {0}$

$int A = A $

$ext A = \{1\}$

$\partial A = \emptyset$

$\bar A = A$
\end{document}