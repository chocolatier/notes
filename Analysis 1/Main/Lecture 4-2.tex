\documentclass[twoside]{article}
\setlength{\oddsidemargin}{0.25 in}
\setlength{\evensidemargin}{-0.25 in}
\setlength{\topmargin}{-0.6 in}
\setlength{\textwidth}{6.5 in}
\setlength{\textheight}{8.5 in}
\setlength{\headsep}{0.75 in}
\setlength{\parindent}{0 in}
\setlength{\parskip}{0.1 in}

%
% ADD PACKAGES here:
%

\usepackage{amsmath,amsfonts,amssymb,graphicx,mathtools,flexisym}

%
% The following commands set up the lecnum (lecture number)
% counter and make various numbering schemes work relative
% to the lecture number.
%
\newcounter{lecnum}
\renewcommand{\thepage}{\thelecnum-\arabic{page}}
\renewcommand{\thesection}{\thelecnum.\arabic{section}}
\renewcommand{\theequation}{\thelecnum.\arabic{equation}}
\renewcommand{\thefigure}{\thelecnum.\arabic{figure}}
\renewcommand{\thetable}{\thelecnum.\arabic{table}}
\newcommand{\N}{\mathbb{N}}
\newcommand{\PS}{\mathcal P}
\newcommand{\Z}{\mathbb{Z}}
\newcommand{\intr}{\text{int }}


%
% The following macro is used to generate the header.
%

%
% Convention for citations is authors' initials followed by the year.
% For example, to cite a paper by Leighton and Maggs you would type
% \cite{LM89}, and to cite a paper by Strassen you would type \cite{S69}.
% (To avoid bibliography problems, for now we redefine the \cite command.)
% Also commands that create a suitable format for the reference list.
\renewcommand{\cite}[1]{[#1]}
\def\beginrefs{\begin{list}%
        {[\arabic{equation}]}{\usecounter{equation}
         \setlength{\leftmargin}{2.0truecm}\setlength{\labelsep}{0.4truecm}%
         \setlength{\labelwidth}{1.6truecm}}}
\def\endrefs{\end{list}}
\def\bibentry#1{\item[\hbox{[#1]}]}

%Use this command for a figure; it puts a figure in wherever you want it.
%usage: \fig{NUMBER}{SPACE-IN-INCHES}{CAPTION}
\newcommand{\fig}[3]{
			\vspace{#2}
			\begin{center}
			Figure \thelecnum.#1:~#3
			\end{center}
	}
% Use these for theorems, lemmas, proofs, etc.
\newtheorem{theorem}{Theorem}[lecnum]
\newtheorem{lemma}[theorem]{Lemma}
\newtheorem{proposition}[theorem]{Proposition}
\newtheorem{claim}[theorem]{Claim}
\newtheorem{corollary}[theorem]{Corollary}
\newtheorem{definition}[theorem]{Definition}
\newtheorem{example}[theorem]{Example}

\newenvironment{proof}{{\bf Proof:}}{\hfill\rule{2mm}{2mm}}

% **** IF YOU WANT TO DEFINE ADDITIONAL MACROS FOR YOURSELF, PUT THEM HERE:

\newcommand\E{\mathbb{E}}
\newcommand\R{\mathbb{R}}


\begin{document}

$(X,d) $ a metric space, $A \subset X$
\begin{definition}
    A set $A$ is open if $A = int A$.
    Equivalently, A is open if $A \subset int A$ (because int $A = A$)

    $int A = \{x : \exists r > 0 , B_r(x) \subset A\}$
\end{definition}

\begin{proposition}
    $\intr A$ is open.  
\end{proposition}

\begin{proof}
    Let $x \in int A$. Then $\exists r > 0$ s.t. $B_r(x) \subset A$.

    ClaimL $B_r(x) \subset int(A)$

    $y \in B_r(x)$.

    $s = d(x,y)$
    $B_{s-r} (y) \subset B_r (x) $ by triangle inequality. 
    $\therefore y$ is an interior point of $A$
    $\therefore B_1(x) \subset int A$
\end{proof}

Note: ext A also open by similar argument. 

\begin{theorem}
    If $A_1, ... A_k$ are open sets, then $\cap_{j=1}^{k}A_j $ is open. 

    If $\{A_i\}_{i \in I}$ is a collection of open sets, then $\cup_{i \in I} A_i$ is open. 

    Proof: Let $x \in \cap_j=1^k A_j$. Then for each $j \exists r_j > 0$ s.t. $B_{r_j} \subset A_j$

    $\therefore B_r(x) \subset B_{r_j}(x) \subset A_j \forall j=1, ..., k$
    $\therefore B_r(x) \subset \cap_{j=1}^{k}A_j$

    ii) If $x \in cup_{i\in I}A_i$ then $x \in A_j$ for some $j \in I$
    $\therefore \exists r> 0 s.t. B_r(x) \subset A_j$
    $\therefore B_r(x) \subset \cup_{i \in I} A_i$

\end{theorem}

\begin{definition}
    A set $A$ is closed if $A^c$ is open. 
\end{definition}

\begin{theorem}
    A set is closed iff $\bar A  = A$
\end{theorem}

\begin{proof}
    $A$ closed $\iff A^c $ open $\iff$ $A^c = int (A^c) \iff A^c = ext (A) \iff int A \cup \partial A = \bar A$.

    Recall $X = int (a) \cup \partial A \cup ext (A)$, but $\bar A = int A \cup \partial A$ as the set is pairwise disjoint. 

    A is closed iff it contains all it's limit points. 
\end{proof}

\begin{theorem}
    i) If $B_1, ..., B_k$ are closed sets, the $\cup_j=1^k is closed.$

    ii) $\cap_{i \in I} B_i$ is closed. 
\end{theorem}

\begin{proof}
    A is closed if $A^c$ is open. $(\cup_i A_i)^c = \cap A_i^c$, $(\cap A_i)^c = \cup A_i^c$
\end{proof}

Note: If A is open, then $\forall x \in A, \exists r_x > 0 s.t. \{y \in X : d(x,y) < r_x\} = B_(r_x) \subset A$.

$$\therefore \bigcup_{x\in A}B_r(x) = A$$

Any open set is a union of open balls. 

$(X,d) $ a metric space. 

If $S \subset X$, then $(S, d_s)$ is a metric space if we define $d_s(x,y) = d(x,y) $ if $x, y \in S$

\begin{proposition}
    Ket $x \in S$. $B_r^S(x) = \{y \in S : d_s(y,x) < r\}$

    $= x \in S$. $B_r^S(x) = \{y \in S : d(y,x) < r\}$

    $= S \cap \{y \in S : d_s(y,x) < r\} = S \subset B_r(x)$
\end{proposition}

Consequence A set $A \subset S$ is open in $S$ iff $\exists $ an open set $U \subset X$ s.t. $A = U \cap S$

$A$ open in $S \implies A = \cup_{x \in A} B_r^s(x) = \cup_{x \in A} S \cap B_r^s(x) =  S \cap  (\cup_{x \in A} B_r^s(x))$

A set $A \subset S$ is closed in $S$ iff $\exists $ a closed set $C \subset X$ s.t. $A = C \cap S$

\end{document}