\documentclass[twoside]{article}
\setlength{\oddsidemargin}{0.25 in}
\setlength{\evensidemargin}{-0.25 in}
\setlength{\topmargin}{-0.6 in}
\setlength{\textwidth}{6.5 in}
\setlength{\textheight}{8.5 in}
\setlength{\headsep}{0.75 in}
\setlength{\parindent}{0 in}
\setlength{\parskip}{0.1 in}

%
% ADD PACKAGES here:
%

\usepackage{amsmath,amsfonts,amssymb,graphicx,mathtools,flexisym}

%
% The following commands set up the lecnum (lecture number)
% counter and make various numbering schemes work relative
% to the lecture number.
%
\newcounter{lecnum}
\renewcommand{\thepage}{\thelecnum-\arabic{page}}
\renewcommand{\thesection}{\thelecnum.\arabic{section}}
\renewcommand{\theequation}{\thelecnum.\arabic{equation}}
\renewcommand{\thefigure}{\thelecnum.\arabic{figure}}
\renewcommand{\thetable}{\thelecnum.\arabic{table}}
\newcommand{\N}{\mathbb{N}}
\newcommand{\PS}{\mathcal P}
\newcommand{\Z}{\mathbb{Z}}
\newcommand{\intr}{\text{int }}
\newcommand{\C}{\mathbb{C}}


%
% The following macro is used to generate the header.
%

%
% Convention for citations is authors' initials followed by the year.
% For example, to cite a paper by Leighton and Maggs you would type
% \cite{LM89}, and to cite a paper by Strassen you would type \cite{S69}.
% (To avoid bibliography problems, for now we redefine the \cite command.)
% Also commands that create a suitable format for the reference list.
\renewcommand{\cite}[1]{[#1]}
\def\beginrefs{\begin{list}%
        {[\arabic{equation}]}{\usecounter{equation}
         \setlength{\leftmargin}{2.0truecm}\setlength{\labelsep}{0.4truecm}%
         \setlength{\labelwidth}{1.6truecm}}}
\def\endrefs{\end{list}}
\def\bibentry#1{\item[\hbox{[#1]}]}

%Use this command for a figure; it puts a figure in wherever you want it.
%usage: \fig{NUMBER}{SPACE-IN-INCHES}{CAPTION}
\newcommand{\fig}[3]{
			\vspace{#2}
			\begin{center}
			Figure \thelecnum.#1:~#3
			\end{center}
	}
% Use these for theorems, lemmas, proofs, etc.
\newtheorem{theorem}{Theorem}[lecnum]
\newtheorem{lemma}[theorem]{Lemma}
\newtheorem{proposition}[theorem]{Proposition}
\newtheorem{claim}[theorem]{Claim}
\newtheorem{corollary}[theorem]{Corollary}
\newtheorem{definition}[theorem]{Definition}
\newtheorem{example}[theorem]{Example}

\newenvironment{proof}{{\bf Proof:}}{\hfill\rule{2mm}{2mm}}

% **** IF YOU WANT TO DEFINE ADDITIONAL MACROS FOR YOURSELF, PUT THEM HERE:

\newcommand\E{\mathbb{E}}
\newcommand\R{\mathbb{R}}
\newcommand\ep{\varepsilon}


\begin{document}

\section{Sequences and Convergence}

$(X,d)$ a metric space. 

A sequences $X$ is a function $f: \N \rightarrow X$.

Usually we write a sequence as $(X_n)_{n=1}^{\infty}$, where $x_n = f(n)$

A subsequence of $(X_n)$ is a sequence $(X_{n_i})_{i=1}^{\infty}$, where $(n_i)_{i=1}^{\infty}$

\begin{definition}
    We say $(X_n)$ converges to $x$ if for any $\varepsilon > 0 \exists N$, s.t. $n > N \implies d(x,x_n) < \varepsilon$.

    Write $X_n \rightarrow x$
\end{definition}

\begin{proposition}
    A sequence in a metric space can have at most one limit. 
\end{proposition}

\begin{proof}
    Suppose $X_n \rightarrow x$, $X_n \rightarrow y$.

    Let $\varepsilon = \frac14d(x,y) > 0$

    Then there exits n such taht $n > N \implies d(x,x_n) < \epsilon$

    $\exists \tilde N$ s.t. $N \tilde N \implies d(y,x_n) < \epsilon$

    $4\varepsilon \leq d(x,y) \leq(x,x_n) + d(x_n,y)$

    $< \varepsilon + \varepsilon$ if $n > max{N, \tilde N}$

    Contradiction. 

    $\therefore x = y$
\end{proof}

\begin{definition}
    A sequence $(x_n)_{n=1}^\infty$ is bounded if the set $\{x_n\}_{n_1}^\infty$ is bounded in $X$.
\end{definition}

E.g. $x_n = 1, \forall N$

$(x_n) = 1,1,1,1,1$

$\{x_n\} = \{1\}$

\begin{proposition}
    A convergent sequence in a metric space $(X,d)$ is bounded. 
\end{proposition}

\begin{proof}
    Suppose $x_n \rightarrow x$. $\exists N$ s.t. $n > N \implies d(x,x_n) < 1$.

    Then at most $x_1, .., x_n$ can lie outside $B_1(x)$

    We can take a large enough radius, and it is bounded. 
\end{proof}

\begin{definition}
    A sequence $(x_n)$ in a metric space $(X,d)$ converges to $x$ if $\forall \ep > 0$, $\exists N s.t. n > N \implies d(x,x_n) < \ep$


    A sequence in $(X,d)$ is Cauchy if $\forall \ep > 0, \exists N$ s.t. $m,n > N \implies d(x_m,x_n) < \ep$

    Same proof shows that every Cauchy sequence is bounded. 
\end{definition}

\begin{theorem}
    If $x_n \rightarrow x$, $y_n \rightarrow y$, then $d(x_n,y_n) \rightarrow d(x,y)$
\end{theorem}

\begin{theorem}
    $d(x,y) \leq d(x,x_n) + d(x_n,y_n) + d(y_n,y)$ by $\triangle$ ineq applied twice. 

    $\implies d(x,y) - d(x_n, y_n) \leq d(x,x_n) + d(y, y_n) \rightarrow 0 as n \rightarrow \infty$

    $d(x_n, y_n) \leq d(x_n, x) + d(x,y) + d(y,y_n)$

    $\implies d(x_n, y_n) - d(x,y) \leq d(x_n,x) + d(y_n, y) \rightarrow 0$ as $n \rightarrow 0$ 
\end{theorem}

\subsection{Sequences in $\R$}

A sequence $(x_n) \subset \R$ is 

increasing (non-decreasing) if $x_n \leq x_{n+1} \forall n$

decreasing (non-increasing) $x_n \geq x_{n+1} \forall n$


strictly increasing if $x_n < x_{n+1} \forall n$

strictly decreasing if $x_n > x_{n+1} \forall n$

A  sequence in $R$ is monotone if it is either increasing or decreasing. 

\begin{theorem}
    Every sequence in $\R$ contains a monotone subsequence. 
\end{theorem}

\begin{proof}
    Let $S = \{k : x_k \geq x_n \forall n \geq k\}$
    
    If $S$ is infinite, the $S = (n_i)_{i=1}^\infty$ with 

    $n_{i+1} > n_i \forall i$

    and $x_{n_{i+1}} \geq x_{n_i} \forall I$

    $\therefore (x_{n_i})_{i=1}^\infty $ is an increasing sequence

    $\therefore $ monotone. 

    Suppose $S$ is finite, say largest element of $S$ is $N$.

    Choose $m_1 > N \implies x_{m_1}$

    Choose $m_2 > m_1 \text{s.t.} x_{m_2} \geq x_{m_1}$ (Such $m_2$ exitsts, if not then $m_1 \in S_1$ not possible)

    Choose $m_3 > m_2 \text{s.t.} x_{m_3} \geq x_{m_2}$

    $\vdots$

    Then $(x_{m_i})$ is an increasing sequence and therefore monotone. 
\end{proof}

\begin{theorem}
    Every bounded monotone sequence in $\R$ has a limit. 
\end{theorem}

\begin{proof}
    See first year notes. 
\end{proof}

\begin{theorem}[Bolzano-Weierstrauss theorem]

    Every bounded sequence of real numbers has a convergent subsequence.
    
\end{theorem}

\begin{proof}
    Any such sequnce has a monotone subsequence. Such a subsequence is also bounded, therefore converegent. 
\end{proof}

\end{document}