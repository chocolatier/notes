\documentclass[twoside]{article}
\setlength{\oddsidemargin}{0.25 in}
\setlength{\evensidemargin}{-0.25 in}
\setlength{\topmargin}{-0.6 in}
\setlength{\textwidth}{6.5 in}
\setlength{\textheight}{8.5 in}
\setlength{\headsep}{0.75 in}
\setlength{\parindent}{0 in}
\setlength{\parskip}{0.1 in}

%
% ADD PACKAGES here:
%

\usepackage{amsmath,amsfonts,amssymb,graphicx,mathtools,flexisym}

%
% The following commands set up the lecnum (lecture number)
% counter and make various numbering schemes work relative
% to the lecture number.
%
\newcounter{lecnum}
\renewcommand{\thepage}{\thelecnum-\arabic{page}}
\renewcommand{\thesection}{\thelecnum.\arabic{section}}
\renewcommand{\theequation}{\thelecnum.\arabic{equation}}
\renewcommand{\thefigure}{\thelecnum.\arabic{figure}}
\renewcommand{\thetable}{\thelecnum.\arabic{table}}
\newcommand{\N}{\mathbb{N}}
\newcommand{\PS}{\mathcal P}
\newcommand{\Z}{\mathbb{Z}}
\newcommand{\intr}{\text{int }}


%
% The following macro is used to generate the header.
%

%
% Convention for citations is authors' initials followed by the year.
% For example, to cite a paper by Leighton and Maggs you would type
% \cite{LM89}, and to cite a paper by Strassen you would type \cite{S69}.
% (To avoid bibliography problems, for now we redefine the \cite command.)
% Also commands that create a suitable format for the reference list.
\renewcommand{\cite}[1]{[#1]}
\def\beginrefs{\begin{list}%
        {[\arabic{equation}]}{\usecounter{equation}
         \setlength{\leftmargin}{2.0truecm}\setlength{\labelsep}{0.4truecm}%
         \setlength{\labelwidth}{1.6truecm}}}
\def\endrefs{\end{list}}
\def\bibentry#1{\item[\hbox{[#1]}]}

%Use this command for a figure; it puts a figure in wherever you want it.
%usage: \fig{NUMBER}{SPACE-IN-INCHES}{CAPTION}
\newcommand{\fig}[3]{
			\vspace{#2}
			\begin{center}
			Figure \thelecnum.#1:~#3
			\end{center}
	}
% Use these for theorems, lemmas, proofs, etc.
\newtheorem{theorem}{Theorem}[lecnum]
\newtheorem{lemma}[theorem]{Lemma}
\newtheorem{proposition}[theorem]{Proposition}
\newtheorem{claim}[theorem]{Claim}
\newtheorem{corollary}[theorem]{Corollary}
\newtheorem{definition}[theorem]{Definition}
\newtheorem{example}[theorem]{Example}

\newenvironment{proof}{{\bf Proof:}}{\hfill\rule{2mm}{2mm}}

% **** IF YOU WANT TO DEFINE ADDITIONAL MACROS FOR YOURSELF, PUT THEM HERE:

\newcommand\E{\mathbb{E}}
\newcommand\R{\mathbb{R}}


\begin{document}

$(X,d)$ is a metric space. 

\begin{definition}
    $X$ is sequentially compact if every sequence in $X$ has a convergent sequence. 
\end{definition}


\begin{definition}
    $X$ is compact if every covering of $X$ has a finite subcover. 
\end{definition}
\begin{theorem}
    X is sequentially compact $\iff$ X compact. 
\end{theorem}

\begin{proof}
    Compactness $\implies$ Sequential Compactness

    Let $(x_n)$ be a sequence in $X$. 

    WTS $(x_w)$ has a convergent subsequence. 

    Let $A = {x_n}$ 
    
    $(x_n) = {x_1, x_2, ....}$

    Case 1 $A$ is finite. 

    Then some value $x_j \in A$ is repeated infinitely often in the sequence $(x_n)$.  Choose the subsequence $ (x_{n_j})$ with $x_{n_j} = \overline x\ \forall j$.

    Then $(x_{n_j}) is a$ constant sequence $\therefore$ convergent. 

    Case 2: A is infinite. 

    Claim: A has a limit point.

    Suppose not. Then $A$ is closed. because $\overline A = A \cup \{\text{limit points}\}$. 
    
    If $a \in A$, then $a$ is not a limit point. 

    $\therefore \exists$ open set $\cup_a$ containing $a$, $A \cap U_a = \{a\}$. 

    $X = A^c \underset{a \in A}\cup U_a$. 

    By compactness $\exists$ a finite subcover of $X$.
    
    $X = A^c \cup U_{a_1} \cup \dots \cup U_{a_n}$. 

    This is a contradiction because $A$ is infinite. 

    Let $x$ be a limit point of $A$. 

    $\forall \varepsilon > 0$, $B_\varepsilon(x)$ contains infinitely points points of $A$ other than $X$.
    
    Choose $x_{n_1} \in A$, s.t. $d(x_{n_1}, x) < 1$.

    Choose  $x_{n_2} \in A$, $n_2 > n$,s.t. $d(x_{n_2}, x) < 1/2$.

    $\vdots$

    Choose  $x_{n_k} \in A$, s.t. $n_{k} > n_{k-1} $ $d(x_{n_k}, x) < 1/k$.


    $\vdots$

    Then $(x_{n_j})_{j=1}^\infty$ is a subsequence of $(x_n)$. $(x_{n_j}) \rightarrow x$ as $j \rightarrow \infty$

    $\therefore (x_n)$ has a convergent subsequence. 

    \medskip

    Sequential compactness $\implies$ Compactness

    Suppose $X$ is sequentially compact. 

    Claim: For each $k \in \N$, $\exists$ finitely many points $\{x_1, \dots, x_N\}$ s.t. $x \in X \implies d(x_i,x) < \frac1k$ for some $x_i \in \{x_1, \dots, x_N\}$

    Choose $x_1$

    Choose $x_2 \in X \backslash B_{\frac1k}(x_1)$

    Choose $x_3 \in X \backslash [B_{\frac1k}(x_2) \cup B_{\frac1k}(x_1)]$


    Keep going until this stop. This must stop. 

    If this does not stop, then $(x_n)$ is a sequence with no convergent subsequence.

    For each $k \in \N$, let $A_k$ be the set chosen by this process. 

    Let $A = \underset{k\in\N}\cup A_k$. Then $A$ is countable, dense.

    \begin{definition}
        $A$ is dense in $X$ if $\overline A = X$.       
    \end{definition}    
 
    Claim: Every open cover of $X$ has a countable subcover. 

    Let $\mathcal F = \{U_i\}_{i \in I}$ be a covering of $X$ by open sets. 

    If $\exists x \in A$, some $r > 0$ s.t. $B_r(x) \subset U_i$ for some $i$, choose one such $U_i$, call it $U_{x,r}$. 
    
    Let $\mathcal{F^*} = \{ U_{x,r} : x \in A, r > 0 \text{ is retained} \}$

    $\mathcal{F^*}$ countable. 
    $\mathcal{F^*}$ covers $X$.

    Let $y \in X$. Then $\exists s > 0$ s.t. $B_s(y) \subset U_i$ for some $i$. 

    Choose rational $r$

    $\frac s4 < r < \frac s2$.

    Choose $x \in A$ s.t. $d(x,y) < r$. 

    Then $y \in B_r(x)$. 

    Then $ y \in B_r(x) \subset B_s(y) \subset U_i$. 

    $\implies \exists U_{x,r} \in \mathcal{F^*} $ s.t. $y \in U_{x,r} \text{ by defintiion of } \mathcal{F^*}$. 

    If $\{U_n\}_{n \in \N}$ is a countable open cover of $X$, then $\exists $ a finite subcover. 

    Let $V_n = {U_1 \cup U_2 \dots \cup U_n}$

    WTS $V_n = X$ for large enough $n$. 

    Suppose not. Then $\forall n \exists x_n \in X \backslash V_n$. 

    $X$ is sequentially compact $\therefore (X_n)$ has a convergent subsequence. 

    By relabelling we may assume $x_n \rightarrow x$. $x \in U_n$ for some $N$. $\therefore x_n \in U_n$, $\forall n$ suff large. $\implies $ contradiction because $x_n \in X \backslash V_n$. 

\end{proof}

\end{document}